\documentclass[11pt,a4paper]{article}

\usepackage[colorlinks=true,pdftex]{hyperref}
\usepackage{latexsym}
\usepackage{amssymb}
\usepackage{amsmath}
\usepackage{amsfonts}
\usepackage{graphicx}
\usepackage{epsfig}
\usepackage{epstopdf}
\usepackage{esint}
\usepackage{hyperref}
\usepackage{amsthm}
\usepackage{mathtools}
\usepackage{color}
\usepackage[english]{babel}
\usepackage[utf8]{inputenc}

\newcommand{\id}{\operatorname{id}}
\newcommand{\R}{\mathbb R}
\newcommand{\Z}{\mathbb Z}
\newcommand{\Q}{\mathbb Q}
\newcommand{\N}{\mathbb N}

\addtolength{\topmargin}{-3.5cm} \addtolength{\oddsidemargin}{-2cm}
\addtolength{\textheight}{+5cm} \addtolength{\textwidth}{+4cm}

\begin{document}
\hrule\hrule
\vspace{1mm}

\noindent {\bf Algebra Advance Course, 2024/25.
\hfill{Assignment 1}}

\vspace{1mm}

 \noindent {\bf Name}: Gonzalo Ortega Carpintero
\vspace{2mm}

\hrule\hrule

\subsection*{Exercise 1}

\subsection*{Exercise 2}
Let $ \Q^* = \Q - \{0\} $ be the multiplicative group of non-zero rationals. Every element $ q \in \Q^* $ can be written as $ q = \frac{a}{b} $ with $ a, b \in \Z $. Every integer admits a prime factorization such that $ a = p_1^{e_1} \dots p_n^{e_n} $, $ b = q_1^{f_1} \dots q_m^{f_m} $ with $p_i, q_i $ prime numbers and $ e_i, f_k \in \N$. If $Q*$ were finitely generated, there would be a finite set $ S $ which generates $Q*$. Each element  $ \frac{a}{b} \in S $ could be decompose into a fraction of prime factorizations. But prime numbers are infinite, so piking a prime $ p $ not included in any of the factorizations of the elements in $ S $ would be a contradiction. As $ p \in \Q^*$ but it can not be generated by elements of $ S $ as it is a prime not in $ S $. Thus, $ \Q^*$ must be infinitely generated.

\subsection*{Exercise 3}

\subsection*{Exercise 4}
Let $ G_1 = \langle a, b \mid a^3 b^5 a^{-3} b^{-5} \rangle $. To show that $ G_1 $ is infinite we will construct a surjective homomorphism from $ G_1 $ to $ \Z $. Define the homomorphism $ \phi \colon G_1  \to \Z $ such that $ \phi(a) = 5 $ and $ \phi(b) = -3 $. From the restrictions of $ G_1 $ we know that $ a^3 b^5 a^{-3} b^{-5} = 1 $, the identity element in $ G_1 $. Indeed,
$$
  \phi (a^3 b^5 a^{-3} b^{-5}) = 3\phi(a) + 5\phi(b) - 3 \phi(a) - 5 \phi(b) = 0,
$$
which is the identity element in $ \Z $ with operation $ + $. For $ a^{n2} b^{n3} $, $ n \in \Z $, we have that $ \phi(a^{n2} b^{n3}) = n $. This makes any element of $ \Z$ reachable from an element of $ G_1 $ by $ \phi $, making $ \phi $ a surjective homomorphism, proving $ G_1 $ is infinite.

Let now $ G_2 = \langle a, b \mid a^2 b^3 \rangle $

\subsection*{Exercise 8}
Let $ G = \langle S \mid R \rangle = F(S) / \langle \langle R \rangle \rangle $ be a finite presentation. All words $ w \in (S \sqcup S^{-1})^* $ such that $ w = 1 $ in $ G $ are the words $ w \in \langle \langle R \rangle \rangle $ by the definition of group presentation. Recall that
$$
  \langle \langle R \rangle \rangle = \bigcup_{i=0}^\infty \left\{ \prod_{j=0}^\infty (g_j^{-1} r_j^{\epsilon_j} g_j) \mid g_j \in G, r_j \in R, \epsilon_j \in \{ \pm 1 \} \right\}.
$$
To enumerate the words $ w $ we can proceed as follows:
\begin{enumerate}
  \item As $ |R| $ is finite, suppose $ |R| = n $. We can enumerate all elements of $ R $ and $ R^{-1} $ numbering them as:
  \begin{align} \label{R}
    r_1, r_1^{-1}, r_2, r_2^{-1}, \dots, r_n, r_n^{-1}.
  \end{align}
  
  \item In the same manner, as $ |S| $ is finite, suppose $ |S| = m $, and enumerate all elements of $ S $ and $ S^{-1} $ as:
  \begin{align} \label{S}
    s_1, s_1^{-1}, s_2, s_2^{-1}, \dots, s_m, s_m^{-1}.
  \end{align}

  \item Finally, now we just need to enumerate the elements of $ \langle \langle R \rangle \rangle $ in a sorted way without enumerating one same element more than once. For so, start enumerating the elements $ g \in F(S) $ by making combinations of the elements of \eqref{S} in a lexicographic order and in increasing word length. As $|S|$ is finite, for each word length $ k $, the amount of words of $ F(S) $ of length $ k $ is going to be $ m ^k $ minus the number of produced words that can be reduced. In any case, there is a finite number of words of length $ k $ in $ F(S) $. Denote this set as  $ F(S)_k $.
  
  For each word length $ k $, we can iterate over the elements of \eqref{R}, and enumerate all the elements
  $$
  \prod_{j=0}^k (g_j^{-1} r_j^{\epsilon_j} g_j) \text{ with } g_j \in F(S)_k, r_j \in R, \epsilon_j \in \{ \pm 1 \}.
  $$

\end{enumerate}

Each $k$-th iteration of Step 3 of the previous procedure is finite as \eqref{R} is finite and $ F(S)_k $ is finite. Therefore, on an input $ w \in (S \sqcup S^{-1})^* $, if $ w = 1 $ in $ G $, as $ w $ would have finite length, our procedure will find it in finite time. Else, our procedure may run forever.

\begin{thebibliography}{9}

  \bibitem{gath}
  Allen Hatcher,
  \textit{Algebraic Topology},
  Allen Hatcher 2001.
  
\end{thebibliography}

\end{document}
