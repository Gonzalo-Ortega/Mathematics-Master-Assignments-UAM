%%%%%%%%%%%%%%%%%%

\documentclass[11pt,a4paper]{article}

%%%%%%%%%%%%%%%%%%

\usepackage{latexsym}
\usepackage{amssymb}
\usepackage{amsmath}
\usepackage{amsfonts}
\usepackage{graphicx}
\usepackage{epsfig}
\usepackage{epstopdf}
\usepackage{esint}
\usepackage{hyperref}
\usepackage{amsthm}

\usepackage{mathtools} %for \stackrel

%%%%%%%%%%%%%%%%%%

\usepackage{color}
\newcommand{\red}{\textcolor{red}}
\newcommand{\blue}{\textcolor{blue}}

%%%%%%%%%%%%%%%%%%

\usepackage[english]{babel}
\usepackage[utf8]{inputenc}

%%%%%%%%%%%%%%%%%%

\addtolength{\topmargin}{-3.5cm} \addtolength{\oddsidemargin}{-2cm}
\addtolength{\textheight}{+5cm} \addtolength{\textwidth}{+4cm}


\parindent=0mm
\parskip 0mm

\thispagestyle{empty}


%%%%%%%%%%%%%%%%%%%%%%%%%%%%%%%%%%%


\begin{document}
\hrule\hrule
\vspace{1mm}

\noindent {\bf Algebraic curves, 2024-25.
\hfill{Sheet 1}}

\vspace{1mm}

 \noindent {\bf Name}: Gonzalo Ortega Carpintero
\vspace{2mm}

\hrule\hrule

\subsubsection*{\bf Exercise 4. Jacobian criterion (Gathmann 2.24)}
\begin{proof}
  Let f be a reduced polynomial and $ P = (a,b) \in V(f) $ a smooth point. We have that
  \begin{gather*}
    (a, b) \text{ is a smooth point } \Leftrightarrow g(x,y) = f(x+a, x+b) \text{ is smooth in } (0,0) \Leftrightarrow \\
    g_1(x,y) \neq 0 \Leftrightarrow c_1x + c_2y \neq 0 \Leftrightarrow c_1 \neq 0 \text{ or } c_2 \neq 0 \Leftrightarrow \\
    \frac{\partial g}{\partial x} = c_1 + \dots \neq 0 \text{ or } \frac{\partial g}{\partial y} = c_2 + \dots \neq 0 \Leftrightarrow \\
    \left(\frac{\partial g}{\partial x}, \frac{\partial g}{\partial y}\right)\Bigg|_{(x,y)=(0,0)} \neq (0,0) \Leftrightarrow \left(\frac{\partial f}{\partial x}, \frac{\partial f}{\partial y}\right)\Bigg|_{(x,y)=(a,b)} \neq (0,0).
  \end{gather*}
\end{proof}

\subsubsection*{\bf Exercise 9. (Gathmann 2.7)}
Let $ f, g \in K[x,y] $ be two reduced polynomials without a common factor, that vanish at $ (0,0) $.

\subsubsection*{\bf (a)}
\begin{proof} 
  Because of Gathmann 1.12, exists $ h(x) \neq 0 \in <f, g>$. We have that $ h = af + bg $ with $ a,b \in \frac{K[x,y]_{(0,0)}}{<f,g>} $ so evaluating at $ 0 $ we have $ h(0) = 0 $ as $ f(0,0) = g (0,0) = 0 $ and can asume that $ h $ has no constant coefficient. We can then factor out some $x^{n_1} $ so that
  $$
    h(x) = x^{n_1} t(x), \text{ with } t(x) = c_1 + c2_x + \dots + c_m x^{m-n_1}.
  $$
  As $ t(x)$ is a unit, we can divide by it having
  $$
    x^{n_1} = \frac{h(x)}{t(x)} \in <f, g>_{K[x, y]_{(0,0)}}.
  $$
  That is to say $ x^{n_1} = 0 $. Analogously $ y^{n_2} = 0 $, and taking $ n= \operatorname{max}(n_1, n_2) $ we have $ x^n = y^n = 0 $.

\end{proof}

\subsubsection*{\bf (b)}
\begin{proof}
  Lets recall the property of power series that says that
$$
  \frac{1}{1-z} = \sum_{i=0}^{\infty} z^i, \text{ for } |z| <1.
$$
Let now $ \frac{f}{g} \in K[x,y]_{(0,0)} $. As $ K $ is a field, we can rewrite $ g = 1 - h $ for some $ h $, and as we are in the local ring around $ (0,0) $, $|h| < 1 $. Thus
$$
\frac{f}{g} = f \frac{1}{1-h} = f \sum_{i=0}^{\infty} h^i,
$$
but because of {\bf (a)}, for some $ n$ all terms of degree grater or equal to $ n $ go to $ 0 $, so the previous sum is actually finite and every element of $\frac{K[x,y]_{(0,0)}}{<f,g>} $ has a polynomial representative.
\end{proof}

\subsubsection*{\bf (c)}
\begin{proof}
  Because of {\bf (b)}, as every element of $\frac{K[x,y]_{(0,0)}}{<f,g>} $ has a polynomial representative there are no infinite term elements in the local ring, therefore
  $$
    \operatorname{dim}_K \frac{K[x,y]_{(0,0)}}{<f,g>} < \infty.
  $$
\end{proof}

\subsubsection*{\bf Exercise 10. (Gathmann 2.8)}
Let $ f, g \in K[x,y] $ be two polynomials that vanish at the origin.

\subsubsection*{\bf (a)}
\begin{proof}
  Let $ f, g $ have no common factor.
  We have that $K[V(g)]_{(0,0)} = \frac{K[x,y]_{(0,0)}}{<g>} $.
  Suppose that there are some $ a_i \neq 0 \in K, i \leq n $ such that
  $
    \sum_i a_i f^i = 0.
  $
  That is
  $
    \sum_i a_i f^i \in <g>,
  $ so for some $ h \neq 0 $ in $ K[x,y]_{(0,0)} $,
  $
    \sum_i a_i f^i = hg
  $, and taking common factor we have
  $$
    \sum_i a_i f^i = f^k (a_k + \sum_i a_i f^{i-k}) = hg.
  $$
  As $ f $ and $ g $ have no common factor there has to be $ h' $ such as that
  $$
    (a_k + \sum_i a_i f^{i-k}) = h'g,
  $$
  but evaluating in $ (0, 0) $ we have
  $$
    (a_k + \sum_i a_i f^{i-k}(0, 0)) = h'(0,0) g(0,0) \Rightarrow (a_k + \sum_i a_i \cdot 0) = h'(0,0) \cdot 0 \Rightarrow a_k = 0,
  $$
  contradicting with $ a_i \neq 0 $ and proving that $\{ f^n: n \in \mathbb Z_{\leq 0}\}$ is linearly independent.
\end{proof}

\subsubsection*{\bf (b)} 
\begin{proof}
  Let $f, g $ have a common factor $ h $ that vanishes at the origin. We can write $ f = f' h^a $ and $ g = g' h^b $. We have
  $$
  \operatorname{dim}_K \frac{K[x,y]_{(0,0)}}{<f, g>} = \operatorname{dim}_K \frac{K[x,y]_{(0,0)}}{<f' h^a, g' h^b>} \geq \operatorname{dim}_K \frac{K[x,y]_{(0,0)}}{<h>},
  $$
  and because of {\bf (a)}, as $ h $ and $ f' $ have no common factors (the same argument could be given taking $ g'$), the infinite family $\{ f'^n: n \in \mathbb Z_{\leq 0}\}$ is linearly independent in $\frac{K[x,y]_{(0,0)}}{<h>} $, so $ \operatorname{dim}_K \frac{K[x,y]_{(0,0)}}{<h>} = \infty $, as polynomials of the form $ f'^n $ are linearly independent spanning an infinite-dimensional vector space. Thus
  $$
  \operatorname{dim}_K \frac{K[x,y]_{(0,0)}}{<f, g>} = \infty.
  $$
\end{proof}


\begin{thebibliography}{9}

  \bibitem{gath}
  Andreas Gathmann,
  \textit{Plane Algebraic Curves},
  Class Notes RPTU Kaiserslautern 2023.
  
\end{thebibliography}

\end{document}
