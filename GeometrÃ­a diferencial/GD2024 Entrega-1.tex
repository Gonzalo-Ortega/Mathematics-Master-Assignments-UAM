%%%%%%%%%%%%%%%%%%

\documentclass[11pt,a4paper]{article}

%%%%%%%%%%%%%%%%%%

\usepackage{latexsym}
\usepackage{amssymb}
\usepackage{amsmath}
\usepackage{amsfonts}
\usepackage{graphicx}
\usepackage{epsfig}
\usepackage{epstopdf}
\usepackage{esint}
\usepackage{hyperref}
\usepackage{amsthm}

\usepackage{mathtools} %for \stackrel

%%%%%%%%%%%%%%%%%%

\usepackage{color}
\newcommand{\red}{\textcolor{red}}
\newcommand{\blue}{\textcolor{blue}}

%%%%%%%%%%%%%%%%%%

\usepackage[spanish]{babel}
\usepackage[utf8]{inputenc}

%%%%%%%%%%%%%%%%%%

\addtolength{\topmargin}{-3.5cm} \addtolength{\oddsidemargin}{-2cm}
\addtolength{\textheight}{+5cm} \addtolength{\textwidth}{+4cm}


\parindent=0mm
\parskip 0mm

\thispagestyle{empty}


%%%%%%%%%%%%%%%%%%%%%%%%%%%%%%%%%%%


\begin{document}
\hrule\hrule
\vspace{1mm}

\noindent {\bf Geometría diferencial, 2024-25.
\hfill{Hoja 1}}

\vspace{1mm}

 \noindent {\bf Nombre}: Gonzalo Ortega Carpintero
\vspace{2mm}

\hrule\hrule

\subsubsection*{\bf Ejercicio 1}
Sean $ A \subset \mathbb R^n $ abierto y $ f: A \rightarrow \mathbb R^n $ continua. El {\it grafo} de $ f $ se define como
$$
  \Gamma_f := \{(x, f(x)): \; x \in A\} \subset \mathbb R^{n+1}. 
$$
La función $ \psi: \Gamma_f \rightarrow \mathbb R^n $, definida como $ \psi(x, f(x)) = x $, es una función continua, puesto que es la función proyección, biyectiva y con inversa continua ya que $ \psi^{-1}(x) = (x, f(x)) $ es continua puesto que $ f $ es continua. Tomando entonces entornos abiertos $ U_i \in \Gamma_f $ con $ \cup U_i = \Gamma_f $ podemos construir el atlas $ \mathcal A = \{ (U_i, \psi) \} $ para ver que $ \Gamma_f $ admite una estructura diferencial.

\subsubsection*{\bf Ejercicio 2}
Sean los conjuntos $ U_i = \{x_i \neq 0\} \in \mathbb R^n $ y $ \bar U_i = \pi(U_i) $ con $ \pi: \mathbb R^{n+1}\backslash \{0\} \rightarrow \mathbb{RP}^n $ la aplicación cociente, y las aplicaciones $ \phi_i :$
\subsubsection*{\bf 1.}
\subsubsection*{\bf 2.}
\subsubsection*{\bf 3.}
\subsubsection*{\bf 4.}
\subsubsection*{\bf 5.}

\end{document}
