%%%%%%%%%%%%%%%%%%

\documentclass[11pt,a4paper]{article}

%%%%%%%%%%%%%%%%%%
\usepackage[colorlinks=true,pdftex]{hyperref}
\usepackage{latexsym}
\usepackage{amssymb}
\usepackage{amsmath}
\usepackage{amsfonts}
\usepackage{graphicx}
\usepackage{epsfig}
\usepackage{epstopdf}
\usepackage{esint}
\usepackage{hyperref}
\usepackage{amsthm}

\usepackage{mathtools} %for \stackrel

%%%%%%%%%%%%%%%%%%

\usepackage{color}
\newcommand{\red}{\textcolor{red}}
\newcommand{\blue}{\textcolor{blue}}

%%%%%%%%%%%%%%%%%%

\usepackage[english]{babel}
\usepackage[utf8]{inputenc}

%%%%%%%%%%%%%%%%%%

\addtolength{\topmargin}{-3.5cm} \addtolength{\oddsidemargin}{-2cm}
\addtolength{\textheight}{+5cm} \addtolength{\textwidth}{+4cm}


\parindent=0mm
\parskip 0mm

\thispagestyle{empty}


%%%%%%%%%%%%%%%%%%%%%%%%%%%%%%%%%%%


\begin{document}
\hrule\hrule
\vspace{1mm}

\noindent {\bf Algebraic curves, 2024-25.
\hfill{Sheet 2}}

\vspace{1mm}

 \noindent {\bf Name}: Gonzalo Ortega Carpintero
\vspace{2mm}

\hrule\hrule

\subsection*{Exercise 3.}

  If $ f \in K[x_0, x_1, x_2] $ be a non-constant homogeneous polynomial and let $ f = \prod_{i=1}^n g_i^{m_i} $ be the decomposition of $ f $ in irreducible polynomials.

\subsubsection*{(a)}
  \begin{proof} \

    \vspace{1mm}
    ($\Rightarrow$)
    If $ V(f) $ is irreducible, suppose that we can not write $ f $ as a power of a irreducible polynomial. Then, there exist $ g_1, g_2 \in K[x_0, x_1, x_2] $, with $ g_1 \neq g_2 $, such that $ f = g_1 g_2 $. But then we have $ V(f) = V(g_1 g_2) = V(g_1) \cup V(g_2) $ as seen in Remark 3.9 of \cite{gath}. This is a contradiction as $ V(f) $ is irreducible so it need to be $ f = g^m $ for some $g$ irreducible.

    \vspace{1mm}
    ($\Leftarrow$)
    If $ f = g^m $, with $ g $ irreducible, then we have $ V(f) = V(g^m) = V(g) $. $ V(g) $ needs to be irreducible, because if not it could be expressed as the union of two curves $ V(g_1) $ and $ V(g_2) $, having $ V(g) = V(g_1) \cap V(g_2) = V(g_1 g_2) $, and it would be $ g =  g_1^m g_2^m $ contradicting the fact of $ g $ being irreducible. Thus, $ V(g^m) $ is irreducible.
  \end{proof}

\subsubsection*{(b)}
  \begin{proof}
    For $ n = 1 $, $ f = g_1^{m_1} $, so $ g_1^{m_1} $ is homogeneous and so $ g_1 $, as $ f $ is homogeneous. Suppose $ h = \prod_{i=1}^{n-1} g_i^{m_i} $ homogeneous. If $ g_n $ was not homogeneous, $ g_n^{m_n} $ neither would be, and the product $ h g_n^{m_n} $ would not be homogeneous as the product of an homogeneous and a non homogeneous polynomials can not be homogeneous. But $ h g_n^{m_n} = \prod_{i=1}^n g_i^{m_i} = f $ contradicting the fact of $ f $ been homogeneous. Thus $ g_n $ must be homogeneous and, because of induction, each $ g_i $ is homogeneous.

    \vspace{1mm}
    As each $ g_i $ is an homogeneous irreducible polynomial, because of {\bf (a)}, $ V(g_i) $ is a irreducible curve. Therefor, $ V(f) = \cup_i V(g_i) $ is a decomposition in irreducible curves of $ V(f) $.
  \end{proof}

\subsection*{Exercise 4}
\begin{proof}
  Let $ I < K[x_0, x_1, x_2] $ be a homogeneous ideal. Let $ K[x_0, x_1, x_2] / I $ be finite dimensional. If $ V(I) \neq \emptyset \subset \mathbb P^2 $ then $ \exists p = [p_0: p_1: p_2] \in \mathbb P^2 $ such that $ \forall f \in I $, $ f(p) = 0 $. Without loos of generality we can asume $ p = [1: p_1: p_2] $ (the argument will follow analogously choosing any other coordinate to be different from $ 0 $). Thus $ \forall n \in \mathbb N, x_0^n \notin I $. 

  \vspace{1mm}
  Therefor we have an infinite family $ \{ x_0^n + I \}_{n \in \mathbb N} $ of elements of $ K[x_0, x_1, x_2] / I $. If the family elements were linearly dependent it would exist $ n_i \in \mathbb N $, and a family of indices $ J \subset \mathbb N $ such that $ x_0^{n_i} = \sum_{j \in J} a_j (x_0^{n_j} + I) $. Let $ g = x_0^{n_i} - \sum_{j \in J} a_j (x_0^{n_j} + I) \in I $. Because of the equivalences of the homogeneous ideal definition seen in class, we have $ g_i \in I $ for each homogeneous part $ g_i $ of $ g $. In particular $ x_0^{n_i} \in I $ forming a contradiction. Thus, $ \{ x_0^n + I \}_{n \in \mathbb N} $ is a linearly independent infinite family of $ K[x_0, x_1, x_2] / I $ so the quotient can not be infinite dimensional contradicting the exercise hypothesis. It then need to be $ V(I) = \emptyset $.
\end{proof}


\begin{thebibliography}{9}

  \bibitem{gath}
  Andreas Gathmann,
  \textit{Plane Algebraic Curves},
  Class Notes RPTU Kaiserslautern 2023.
  
\end{thebibliography}


\end{document}
