%%%%%%%%%%%%%%%%%%

\documentclass[11pt,a4paper]{article}

%%%%%%%%%%%%%%%%%%
\usepackage[colorlinks=true,pdftex]{hyperref}
\usepackage{latexsym}
\usepackage{amssymb}
\usepackage{amsmath}
\usepackage{amsfonts}
\usepackage{graphicx}
\usepackage{epsfig}
\usepackage{epstopdf}
\usepackage{esint}
\usepackage{hyperref}
\usepackage{amsthm}

\usepackage{mathtools} %for \stackrel

%%%%%%%%%%%%%%%%%%

\usepackage{color}
\newcommand{\red}{\textcolor{red}}
\newcommand{\blue}{\textcolor{blue}}

%%%%%%%%%%%%%%%%%%

\usepackage[english]{babel}
\usepackage[utf8]{inputenc}

%%%%%%%%%%%%%%%%%%

\addtolength{\topmargin}{-3.5cm} \addtolength{\oddsidemargin}{-2cm}
\addtolength{\textheight}{+5cm} \addtolength{\textwidth}{+4cm}


\parindent=0mm
\parskip 0mm

\thispagestyle{empty}


%%%%%%%%%%%%%%%%%%%%%%%%%%%%%%%%%%%


\begin{document}
\hrule\hrule
\vspace{1mm}

\noindent {\bf Algebraic curves, 2024-25.
\hfill{Sheet 2}}

\vspace{1mm}

 \noindent {\bf Name}: Gonzalo Ortega Carpintero
\vspace{2mm}

\hrule\hrule

\subsection*{Exercise 1.}
\begin{proof}
  Let $ X $ be a set, and let $ U_i $ for $ i \in I $ be topological spaces, where $ X = \bigcup U_i $. Let $ \tau_i $ be the topology of each $ U_i $. For every $ i, j \in I $, the topologies $ \tau_i, \tau_j $ on $ U_i $ and $ U_j $ restricted to $ U_i \cap U_j $ agree. A definition of restricted topology, also known as the relative topology can be found in Section 6 of \cite{will}. We can define a topology over $ X $ by considering the union of all open sets of all $ U_i$'s. That is 
  $$
    \mathcal T = \{ V \subseteq X : V \cap U_i \in \tau_i, \; \forall i \in I\}.
  $$
  Let's check that $ \mathcal T $ is actually a topology. We have $ \emptyset \in \mathcal T $ as for all $i \in I $, $\emptyset \cap U_i = \emptyset \in \tau_i $. The same happens with $ X $, as for all $ i\in I$, $X \cap U_i = U_i \in \tau_i$. Therefore $ \emptyset \in \mathcal T $ and $ X \in \mathcal T $.

  \vspace{1mm}
  Suppose $ V_1, V_2 \in \mathcal T $, then
  $$
    (V_1 \cap V_2) \cap U_i = (V_1\cap U_i) \cap (V_2 \cap U_i) \in \tau_1,
  $$
  as $ V_1\cap U_i $ and $ V_2\cap U_i $ are open sets of $U_i$, and the intersection of open sets is open. Thus $ V_1 \cap V_2 \in \mathcal T $.

  \vspace{1mm}
  In the same manner, if $ \{V_\alpha\}_{\alpha \in A}$ is a collection of sets of $ \mathcal T $, we have
  $$
    \left(\bigcup_\alpha V_\alpha \right) = \bigcup_\alpha (V_\alpha \cup U_i) \in \tau_i
  $$
  as each $ V_\alpha \cup U_i $ is open in $U_i$. We have then $ \bigcup_\alpha V_\alpha \in \mathcal T $, and we can conclude that $ \mathcal T $ is a topology of X.

  \vspace{1mm}
  We have proved existence, lets prove now the uniqueness of $ \mathcal T $. Suppose we have two topologies $ \mathcal T_1 $ and $ \mathcal T_2 $ of $ X $ that restrict to the topology of each $ U_i $. Let $ \tau_i $ be the restricted topology of $ \mathcal T_1 $ on each $ U_i $ and $ \tau_j $ be the restricted topology of $ \mathcal T_2 $ on each $ U_j $. And let $ \tau_i', \tau_j' $ be the restricted topologies of $ \tau_i $ and $ \tau_j $ on $ U_i \cap U_j $. We then have that 
  $$
    V \in \mathcal T_1 \Leftrightarrow V \cap U_i \in \tau_i \Leftrightarrow V \cap U_i \cap U_j \in \tau_i' \underset{hypothesis}{\Leftrightarrow} V \cap U_i \cap U_j \in \tau_j' \Leftrightarrow  V \cap U_j \in \tau_j \Leftrightarrow V \in \mathcal T_2.
  $$
  Therefore, $ T_1 = T_2 $, so there is exactly one topology on $ X $ that restricts to the topology on each $ U_i $.
\end{proof}
\subsection*{Exercise 3.}

  If $ f \in K[x_0, x_1, x_2] $ be a non-constant homogeneous polynomial and let $ f = \prod_{i=1}^n g_i^{m_i} $ be the decomposition of $ f $ in irreducible polynomials.

\subsubsection*{(a)}
  \begin{proof} \

    \vspace{1mm}
    ($\Rightarrow$)
    If $ V(f) $ is irreducible, suppose that we can not write $ f $ as a power of a irreducible polynomial. Then, there exist $ g_1, g_2 \in K[x_0, x_1, x_2] $, with $ g_1 \neq g_2 $, such that $ f = g_1 g_2 $. But then we have $ V(f) = V(g_1 g_2) = V(g_1) \cup V(g_2) $ as seen in Remark 3.9 of \cite{gath}. This is a contradiction as $ V(f) $ is irreducible so it need to be $ f = g^m $ for some $g$ irreducible.

    \vspace{1mm}
    ($\Leftarrow$)
    If $ f = g^m $, with $ g $ irreducible, then we have $ V(f) = V(g^m) = V(g) $. $ V(g) $ needs to be irreducible, because if not it could be expressed as the union of two curves $ V(g_1) $ and $ V(g_2) $, having $ V(g) = V(g_1) \cup V(g_2) = V(g_1 g_2) $, and it would be $ g =  g_1^m g_2^m $ contradicting the fact of $ g $ being irreducible. Thus, $ V(g^m) $ is irreducible.
  \end{proof}

\subsubsection*{(b)}
  \begin{proof}
    For $ n = 1 $, $ f = g_1^{m_1} $, so $ g_1^{m_1} $ is homogeneous and so $ g_1 $, as $ f $ is homogeneous. Suppose $ h = \prod_{i=1}^{n-1} g_i^{m_i} $ homogeneous. If $ g_n $ was not homogeneous, $ g_n^{m_n} $ neither would be, and the product $ h g_n^{m_n} $ would not be homogeneous as the product of an homogeneous and a non homogeneous polynomials can not be homogeneous. But $ h g_n^{m_n} = \prod_{i=1}^n g_i^{m_i} = f $ contradicting the fact of $ f $ been homogeneous. Thus $ g_n $ must be homogeneous and, because of induction, each $ g_i $ is homogeneous.

    \vspace{1mm}
    As each $ g_i $ is an homogeneous irreducible polynomial, because of {\bf (a)}, $ V(g_i) $ is a irreducible curve. Therefore, $ V(f) = \cup_i V(g_i) $ is a decomposition in irreducible curves of $ V(f) $.
  \end{proof}

\subsection*{Exercise 4.}
\begin{proof}
  Let $ I < K[x_0, x_1, x_2] $ be a homogeneous ideal. Let $ K[x_0, x_1, x_2] / I $ be finite dimensional. If $ V(I) \neq \emptyset \subset \mathbb P^2 $ then $ \exists p = [p_0: p_1: p_2] \in \mathbb P^2 $ such that $ \forall f \in I $, $ f(p) = 0 $. Without lose of generality we can asume $ p = [1: p_1: p_2] $ (the argument will follow analogously choosing any other coordinate to be different from $ 0 $). Thus $ \forall n \in \mathbb N, x_0^n \notin I $. 

  \vspace{1mm}
  Therefore we have an infinite family $ \{ x_0^n + I \}_{n \in \mathbb N} $ of elements of $ K[x_0, x_1, x_2] / I $. If the family elements were linearly dependent it would exist $ n_i \in \mathbb N $ and a finite family of indices $ J \subset \mathbb N $ such that $ x_0^{n_i} = \sum_{j \in J} a_j (x_0^{n_j} + I) $. Let $ g = x_0^{n_i} - \sum_{j \in J} a_j (x_0^{n_j} + I) \in I $. Because of the equivalences of the homogeneous ideal definition seen in class, we have $ g_i \in I $ for each homogeneous part $ g_i $ of $ g $. In particular $ x_0^{n_i} \in I $ forming a contradiction. Thus, $ \{ x_0^n + I \}_{n \in \mathbb N} $ is a linearly independent infinite family of $ K[x_0, x_1, x_2] / I $ so the quotient can not be infinite dimensional contradicting the exercise hypothesis. It then need to be $ V(I) = \emptyset $.
\end{proof}

\subsection*{Exercise 8 (Gathmann 4.9)}

\subsubsection*{(a)}
Over $ \mathbb C $, let $ f = x + y^2 $ and $ g = x + y^2 - x^3 $. Their respective homogenizations are $ f^{h} = xz +y^2 $ and $ g^{h} = xz^2 + y^2z - x^3 $. Making $ z = 0 $ to obtain their points at infinity we get $ f^{h}(z=0) = y^2 $ and $ g^{h}(z=0) = -x^3 $. So the only point at infinity of $ f $ is $ (1:0:0) $ and of $ g $, $ (0:1:0) $. Thus, $ V(f) $ and $ V(g) $ do not intersect at infinity. In the affine part, their only intersection point is $ p = (0:0:1) $, and as $ \mathbb C = \bar{\mathbb C}$, because of Bézout's theorem
$$
  \mu_p(f, g) = \deg f \cdot \deg g = 2 \cdot 3 = 6.
$$

\subsubsection*{(b)}
Over $ \mathbb C $, now let $ f = y^2 - x_2 $ and $ g = (x + y + 1) (y - x + 1) = y^2 - x^2 + 2y + 1 $. Their homogenizations are $ f^h = y^2 - x^2 + z^2 $ and $ g^h = y^2 - x^2 + 2yz + z^2 $. Then we have $ f^h(z=0) = y^2 - x^2 $ and $ g^h(z=0) = y^2 - x^2 $, so both curves have two common points at infinity, $ p_1 = (1:1:0) $ and $ p_2 = (1:-1:0) $. In the affine part we have other two common points, $ p_3 = (1:1:1) $ and $ p_4 = (1:-1:1) $ so using Bézout's theorem we have
$$
  \sum_{i=1}^{4} \mu_{p_i}(f, g) = \deg f \cdot \deg g = 2 \cdot 2 = 4,
$$
so it could only be $ \mu_{p_i}(f, g) = 1 $ for each $ i $.

\begin{thebibliography}{9}

  \bibitem{gath}
  Andreas Gathmann,
  \textit{Plane Algebraic Curves},
  Class Notes RPTU Kaiserslautern 2023.

  \bibitem{will}
  Stephen Willard,
  \textit{General Topology},
  Addison-Wesley 1970.
  
\end{thebibliography}


\end{document}
