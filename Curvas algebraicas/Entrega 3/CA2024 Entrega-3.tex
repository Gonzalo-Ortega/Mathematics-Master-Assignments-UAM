%%%%%%%%%%%%%%%%%%

\documentclass[11pt,a4paper]{article}

%%%%%%%%%%%%%%%%%%
\usepackage[colorlinks=true,pdftex]{hyperref}
\usepackage{latexsym}
\usepackage{amssymb}
\usepackage{amsmath}
\usepackage{amsfonts}
\usepackage{graphicx}
\usepackage{epsfig}
\usepackage{epstopdf}
\usepackage{esint}
\usepackage{hyperref}
\usepackage{amsthm}

\usepackage{mathtools} %for \stackrel

%%%%%%%%%%%%%%%%%%

\usepackage{color}
\newcommand{\red}{\textcolor{red}}
\newcommand{\blue}{\textcolor{blue}}

%%%%%%%%%%%%%%%%%%

\usepackage[english]{babel}
\usepackage[utf8]{inputenc}

%%%%%%%%%%%%%%%%%%

\addtolength{\topmargin}{-3.5cm} \addtolength{\oddsidemargin}{-2cm}
\addtolength{\textheight}{+5cm} \addtolength{\textwidth}{+4cm}

%%%%%%%%%%%%%%%%%%%%%%%%%%%%%%%%%%%


\begin{document}
\hrule\hrule
\vspace{1mm}

\noindent {\bf Algebraic curves, 2024-25.
\hfill{Sheet 3}}

\vspace{1mm}

 \noindent {\bf Name}: Gonzalo Ortega Carpintero
\vspace{2mm}

\hrule\hrule

\subsection*{Exercise 4 (Gathmann 5.7.).}
\subsubsection*{(a)}
\begin{proof}
  Let $ F = F_1 \cdots F_n $ be a reduced curve of degree 
  $ d $ with its decomposition into $ F_i $ irreducible components. Lets proof the statement by induction over the number $ n $ of irreducible components. For $ n = 1 $, $ F = F_1 $ is an irreducible curve, so using \cite[Proposition 5.6]{gath}, we have that $ F $ has at most $ \binom{d - 1}{2} \leq \binom{d}{2} $ singular points. Assuming the hypothesis for $ n = m $, lets check it for $ n = m+1$. We have $ F = F_1 \cdots F_m F_{m+1} $, were $ \deg F = d $ and $ \deg F_1 \cdots F_m = d - \deg F_{m+1} $. Denote $d' = \deg F_{m+1} $. Using the induction hypothesis we know that $ F_1 \dots F_m $ has at most $ \binom{d-d'}{2} $ singular points and, because \cite[Proposition 5.6]{gath}, $ F_{m+1} $ has $ \binom{d-1}{2} $. The  Bézout's theorem says that the intersection multiplicity of both curves is $ (d-d')d' $ so the singular points $ F $ has are at most
  \begin{align*}
    \binom{d-d'}{2} + \binom{d' - 1}{2} +(d-d')d' &= \frac{(d-d')(d-d'-1) + (d'-1)(d'-2) + 2dd' - 2d'^2}{2} \\
    &= \frac{d(d-1)}{2} - d' +1 \leq \frac{d(d-1)}{2} = \binom{d}{2}
  \end{align*}
  noting that $ 1 \leq d' $.
\end{proof}

\subsubsection*{(b)}
To find an example for each $ d $ where $ F $ has exactly $ \binom{d}{2} $ singular points, its enough to take $ d $ linear independent lines $ L_i $, $ i \in [1, d] $ where for each $ i \neq j \in [1, d] $, if $ P = L_i \cup L_j $, then $ L_i \cup L_k \neq P, \ \forall k \neq i,j \in [1, d] $. A concrete example could be formed by the lines of the form $ \left\{d(x+d)\right\}_{d \in \mathbb N} $. (Considering that the characteristic of the field $ K $ is not a multiple of $ d $).

\subsection*{Exercise 7 (Gathmann 6.21.).}
\begin{proof}
We define, following \cite[Construction 3.13]{gath}, the ring homomorphism between the coordinate ring $ A(F) = K[x, y] / \langle F \rangle $ and the homogeneous coordinate ring of degree $ d $, $ S(F)$, formed by the homogeneous elements of degree $ d $ of $ S(F) = K[x, y, z] / \langle F \rangle$, as
  \begin{align*}
    \phi: A(F) &\longrightarrow S_d(F) \\
    f^i = \sum_{i+j \leq d} a_{i,j} x^i y^j &\longmapsto \phi (f^i) = f^h = \sum_{i+j \leq d} a_{i,j} x^i y^j z^{d-i-j}.
  \end{align*}
  If $f, g \in A(F) $, $\phi $ and $ d = \operatorname{max}\{\operatorname{deg}f, \operatorname{deg}g\} $, $\phi$ is an homomorphism as
  \begin{align*}
    \phi(1) &= \phi(1 x^0 y^0) = 1 x^0 y^0 z^0 = 1, \\
    \phi(f + g) &= \phi \left(\sum_{i + j \leq d} a_{i,j} x^i y^j + \sum_{i + j \leq d} b_{i,j} x^i y^j \right) = \phi \left(\sum_{i + j \leq d} (a_{i,j} + b_{i,j})  x^i y^j \right) \\
    &= \sum_{i + j \leq d} (a_{i,j} + b_{i,j})  x^i y^j z^{d-i-j} = \sum_{i + j \leq d} a_{i,j}  x^i y^j z^{d-i-j} + \sum_{i + j \leq d} b_{i,j}  x^i y^j z^{d-i-j} \text{ and} \\
    &= \phi(f) + \phi(g)\\ 
  \end{align*}

  \begin{align*}
    \phi(f \cdot g) &= \phi\left(\sum_{i + j \leq d} a_{i,j} x^i y^j \cdot \sum_{k + l \leq d} b_{k,l} x^k y^l\right) = \phi\left(\sum_{\substack{i + j \leq d\\k + l \leq d}} a_{i,j} b_{k,l} x^{i+k} y^{j+l} \right) \\
    &= \sum_{\substack{i + j \leq d\\k + l \leq d}} a_{i,j} b_{k,l} x^{i+k} y^{j+l} z^{2d-i-j-k-l} = \sum_{i + j \leq d} a_{i,j} x^{i} y^{j} z^{d-i-j} \cdot \sum_{k + l \leq d} b_{k,l} x^{k} y^{l} z^{d-k-l} \\
    &= \phi(f) \cdot \phi(g).
  \end{align*}
  Actually, $ \phi $ is an isomorphism, as we can define its inverse as
  \begin{align*}
    \phi^{-1}: S_d(F) &\longrightarrow A(F) \\
    f^h = \sum_{i+j+k = d} a_{i,j,k} x^i y^j z^k &\longmapsto \phi^{-1} (f^h) = f^i = \sum_{i+j \leq d} a_{i,j,k} x^i y^j.
  \end{align*}
  Now, we can define the desired isomorphism between $ K(F) = \left\{\frac{f}{g}: f, g \in A(F)\right\} $ and $ K(F^h) = $ $ \left\{\frac{f}{g}: f, g \in S_d(F)\right\} $ as
  \begin{align*}
    \Phi: K(F) &\longrightarrow K(F^h) \\
    \frac{f^i}{g^i} &\longmapsto \Phi \left(\frac{f^i}{g^i}\right)= \frac{\phi(f^i)}{\phi(g^i)} = \frac{f^h}{g^h},
  \end{align*}
  witch has inverse
  \begin{align*}
    \Phi^{-1}: K(F^h) &\longrightarrow K(F) \\
    \frac{f^h}{g^h} &\longmapsto \Phi^{-1} \left(\frac{f^h}{g^h}\right)= \frac{\phi^{-1}(f^h)}{\phi^{-1}(g^h)} = \frac{f^i}{g^i},
  \end{align*}
  and, given $ \frac{f}{g}, \frac{h}{k} \in K(F) $, verifies
  \begin{align*}
    \Phi(\frac{1}{1}) &= \frac{\phi(1)}{\phi(1)} = \frac{1}{1}, \\
    \Phi(\frac{f}{g} + \frac{h}{k}) &= \Phi(\frac{fk + gh}{gk}) = \frac{\phi(fk + gh)}{\phi(gk)} = \frac{\phi(f) \cdot \phi(k) + \phi(g) \cdot\phi(h)}{\phi(g) \cdot \phi(k)} = \frac{\phi(f)}{\phi(g)} + \frac{\phi(h)}{\phi(k)} \\
    &= \Phi(\frac{f}{g}) + \Phi(\frac{h}{k}), \\
    \Phi(\frac{f}{g} \cdot \frac{h}{k}) &= \Phi(\frac{f\cdot h}{g \cdot k}) = \frac{\phi(f\cdot h)}{\phi(g \cdot k)} = \frac{\phi(f) \cdot \phi(h)}{\phi(g) \cdot \phi(k)} = \Phi(\frac{f}{g})\cdot \Phi(\frac{h}{k}).
  \end{align*}
\end{proof}

\newpage
\subsection*{Exercise 8 (Gathmann 6.25.).}
Let $ F = y^2 z - x^3 + xz^2 $ and $ \varphi = \frac{y}{z} $. $ F = 0$ at the points $ P_1 = (0:0:1) $, $ P_2 = (1:0:1) $, $ P_3 = (-1:0:1) $, $ P_4 = (0:1:0) $. Hence, using \cite[Construction 6.17]{gath} and \cite[Algorithm 2.12]{gath}, we compute the multiplicity at each $ P_i $ of $ \varphi $ at $ F $.
\begin{align*}
  \mu_{P_1}(y) &= \mu_{(0,0)}(y, y^2-x^3+x) = \mu_{(0,0)}(y, x(1-x^2)) = 1 \\
  \mu_{P_1}(z) &= \mu_{(0,0)}(1, y^2-x^3+x) = 0 \\
  \mu_{P_1}(\varphi) &= \mu_{P_1}(y) - \mu_{P_1}(z) = 1 - 0 = 1 \\
  \\
  \mu_{P_2}(y) &= \mu_{(1,0)}(y, y^2-x^3+x) = \mu_{(0,0)}(y, y^2-(x+1)^3+x+1) = \mu_{(0,0)}(y, -x(x^2+3x-2)) = 1 \\
  \mu_{P_2}(z) &= \mu_{(1,0)}(1, y^2-x^3+x) = \mu_{(0,0)}(1, y^2-(x+1)^3+x+1) = 0 \\
  \mu_{P_2}(\varphi) &= \mu_{P_2}(y) - \mu_{P_2}(z) = 1 - 0 = 1 \\
  \\
  \mu_{P_3}(y) &= \mu_{(-1,0)}(y, y^2-x^3+x) = \mu_{(0,0)}(y, y^2-(x-1)^3+x-1) = \mu_{(0,0)}(y, x(x^2-3x+4)) = 1 \\
  \mu_{P_3}(z) &= \mu_{(-1,0)}(1, y^2-x^3+x) = \mu_{(0,0)}(1, y^2-(x-1)^3+x-1) = 0 \\
  \mu_{P_3}(\varphi) &= \mu_{P_3}(y) - \mu_{P_3}(z) = 1 - 0 = 1 \\
  \\
  \mu_{P_4}(y) &= \mu_{(0,0)}(1, z-x^3+xz^2) = 0 \\
  \mu_{P_4}(z) &= \mu_{(0,0)}(z, z-x^3+xz^2) = \mu_{(0,0)}(z, z(1+xz)-x^3) = \mu_{(0,0)}(z, -x^3) = 3\\
  \mu_{P_4}(\varphi) &= \mu_{P_4}(y) - \mu_{P_4}(z) = 0 - 3 = -3 \\
\end{align*}
Now, following \cite[Construction 6.23]{gath}, we have
$$
  \operatorname{div}\frac{y}{z} = 1 \cdot (0:0:1) + 1 \cdot (1:0:1) + 1 \cdot (-1:0:1) - 3 \cdot (0:1:0).
$$

\begin{thebibliography}{9}

  \bibitem{gath}
  Andreas Gathmann,
  \textit{Plane Algebraic Curves},
  Class Notes RPTU Kaiserslautern 2023.
  
\end{thebibliography}

\end{document}
