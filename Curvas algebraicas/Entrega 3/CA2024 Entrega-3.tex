%%%%%%%%%%%%%%%%%%

\documentclass[11pt,a4paper]{article}

%%%%%%%%%%%%%%%%%%
\usepackage[colorlinks=true,pdftex]{hyperref}
\usepackage{latexsym}
\usepackage{amssymb}
\usepackage{amsmath}
\usepackage{amsfonts}
\usepackage{graphicx}
\usepackage{epsfig}
\usepackage{epstopdf}
\usepackage{esint}
\usepackage{hyperref}
\usepackage{amsthm}

\usepackage{mathtools} %for \stackrel

%%%%%%%%%%%%%%%%%%

\usepackage{color}
\newcommand{\red}{\textcolor{red}}
\newcommand{\blue}{\textcolor{blue}}

%%%%%%%%%%%%%%%%%%

\usepackage[english]{babel}
\usepackage[utf8]{inputenc}

%%%%%%%%%%%%%%%%%%

\addtolength{\topmargin}{-3.5cm} \addtolength{\oddsidemargin}{-2cm}
\addtolength{\textheight}{+5cm} \addtolength{\textwidth}{+4cm}


\parindent=0mm
\parskip 0mm

\thispagestyle{empty}


%%%%%%%%%%%%%%%%%%%%%%%%%%%%%%%%%%%


\begin{document}
\hrule\hrule
\vspace{1mm}

\noindent {\bf Algebraic curves, 2024-25.
\hfill{Sheet 3}}

\vspace{1mm}

 \noindent {\bf Name}: Gonzalo Ortega Carpintero
\vspace{2mm}

\hrule\hrule

\subsection*{Exercise 8 (Gathmann 6.25.).}
Let $ F = y^2 z - x^3 + xz^2 $ and $ \varphi = \frac{y}{z} $. $ F = 0$ at the points $ P_1 = (0:0:1) $, $ P_2 = (1:0:1) $, $ P_3 = (-1:0:1) $, $ P_4 = (0:1:0) $. Hence, using \cite[Construction 6.17]{gath} and \cite[Algorithm 2.12]{gath}, we compute the multiplicity at each $ P_i $ of $ \varphi $ at $ F $.
\begin{align*}
  \mu_{P_1}(y) &= \mu_{(0,0)}(y, y^2-x^3+x) = \mu_{(0,0)}(y, x(1-x^2)) = 1 \\
  \mu_{P_1}(z) &= \mu_{(0,0)}(1, y^2-x^3+x) = 0 \\
  \mu_{P_1}(\varphi) &= \mu_{P_1}(y) - \mu_{P_1}(z) = 1 - 0 = 1 \\
  \\
  \mu_{P_2}(y) &= \mu_{(1,0)}(y, y^2-x^3+x) = \mu_{(0,0)}(y, y^2-(x+1)^3+x+1) = \mu_{(0,0)}(y, -x(x^2+3x-2)) = 1 \\
  \mu_{P_2}(z) &= \mu_{(1,0)}(1, y^2-x^3+x) = \mu_{(0,0)}(1, y^2-(x+1)^3+x+1) = 0 \\
  \mu_{P_2}(\varphi) &= \mu_{P_2}(y) - \mu_{P_2}(z) = 1 - 0 = 1 \\
  \\
  \mu_{P_3}(y) &= \mu_{(-1,0)}(y, y^2-x^3+x) = \mu_{(0,0)}(y, y^2-(x-1)^3+x-1) = \mu_{(0,0)}(y, x(x^2-3x+4)) = 1 \\
  \mu_{P_3}(z) &= \mu_{(-1,0)}(1, y^2-x^3+x) = \mu_{(0,0)}(1, y^2-(x-1)^3+x-1) = 0 \\
  \mu_{P_3}(\varphi) &= \mu_{P_3}(y) - \mu_{P_3}(z) = 1 - 0 = 1 \\
  \\
  \mu_{P_4}(y) &= \mu_{(0,0)}(1, z-x^3+xz^2) = 0 \\
  \mu_{P_4}(z) &= \mu_{(0,0)}(z, z-x^3+xz^2) = \mu_{(0,0)}(z, z(1+xz)-x^3) = \mu_{(0,0)}(z, -x^3) = 3\\
  \mu_{P_4}(\varphi) &= \mu_{P_4}(y) - \mu_{P_4}(z) = 0 - 3 = -3 \\
\end{align*}
Now, following \cite[Construction 6.23]{gath}, we have
$$
  \operatorname{div}\frac{y}{z} = 1 \cdot (0:0:1) + 1 \cdot (1:0:1) + 1 \cdot (-1:0:1) - 3 \cdot (0:1:0).
$$

\begin{thebibliography}{9}

  \bibitem{gath}
  Andreas Gathmann,
  \textit{Plane Algebraic Curves},
  Class Notes RPTU Kaiserslautern 2023.
  
\end{thebibliography}

\end{document}
