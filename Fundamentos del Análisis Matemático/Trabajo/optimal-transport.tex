\section{Optimal transport}
The main result of optimal transport theory is the solution of Kantorovich's problem for general costs, the existence of an optimal transport plan.

\begin{proposition}
    Let $ c: X \times Y \to [0, \infty] $ be lower semicontinous, and let $ \mu \in \mathcal P(X) $ and $ \nu \in \mathcal P(Y) $. Then there exists a coupling $ \bar \gamma \in \Gamma(\mu, \nu) $ that verifies
    $$
        \bar \gamma = \min \left\{ \gamma \in \Gamma(\mu, \nu) \ : \ \int_{X \times Y} c(x, y) d\gamma(x,y) \right\}.
    $$
\end{proposition}
\begin{proof}
    (to do)
\end{proof}

We will denote the set of probability measures over a space X by $ \mathcal P(X) $. 

\begin{example}[Mean and variance in $ \mathbb R $]
    
\end{example}

\begin{definition}
    Let $ (X, d) $ be a locally compact and separable, metric space. Let $ 1 \leq p < \infty $. The {\it set of probability measures with finite $p$-moment} is defined As
    $$
        \mathcal P_p (X) := \left\{ \sigma \in \mathcal P (X) : \int_X d(x, x_0)^p d \mu (x) < \infty \text{ for some } x_0 \in X \right\}.
    $$
\end{definition}

\begin{proposition}
    The definition of $ \mathcal P_p (X) $ is independent of the base point $ x_0 $
\end{proposition}
\begin{proof}
    (to do)
\end{proof}

\begin{definition}[$p$-Wasserstein distance]
    Given $ u, v \in \mathcal P_p (X) $, the {\it $p$-Wasserstein distance} is defined as
    $$
        W_p(u, v) := \left( \inf_{\gamma \in \Gamma(u, v)} \int_{X \times X} d(x,y)^p d\gamma(x, y)\right)^{\frac{1}{p}}.
    $$
\end{definition}

\begin{proposition}
    $W_p$ is a distance on the space $ \mathcal P_p(X) $.
\end{proposition}

\begin{proof}
    We will follow the steps made in \cite{Figalli}[Theorem 3.1.5].
    To prove the triangle inequality, let $ \mu_1, \mu_2, \mu_3 \in \mathcal P_p(X) $ and 

    (to do)
\end{proof}