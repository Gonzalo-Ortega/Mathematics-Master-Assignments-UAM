\documentclass[12pt, a4paper]{article}

\usepackage[a4paper,bindingoffset=3mm,bottom=35mm]{geometry}
\usepackage[colorlinks=true, linkcolor=blue, citecolor=blue, pdftex]{hyperref}

\usepackage[pdftex]{graphicx}  
\usepackage[english]{babel}
\usepackage[utf8]{inputenc}
\usepackage{amsmath,amssymb,amsthm}

\newtheorem{theorem}{Theorem}[section]
\newtheorem{proposition}[theorem]{Proposition}
\newtheorem{lemma}[theorem]{Lemma}
\newtheorem{corolary}[theorem]{Corolary}
\newtheorem{fact}[theorem]{Fact}

\theoremstyle{definition}
\newtheorem{example}[theorem]{Example}
\newtheorem{definition}[theorem]{Definition}
\newtheorem{remark}[theorem]{Remark}

\newcommand{\authorname}{Gonzalo Ortega Carpintero}
\newcommand{\institution}{Universidad Autónoma de Madrid}
\newcommand{\projecttitle}{Optimal transport for

Topological Data Analysis}

\newcommand{\dgmp}{\operatorname{Dgm}_p}
\newcommand{\dgmi}{\operatorname{Dgm}_\infty}
\newcommand{\costp}{\operatorname{cost}_p}
\newcommand{\costi}{\operatorname{cost}_\infty}
\newcommand{\wdp}{\omega_p}
\newcommand{\wdi}{\omega_\infty}
\newcommand{\twdp}{\tilde \omega_p}
\newcommand{\twdi}{\tilde \omega_\infty}
\newcommand{\p}{\mathcal P}
\newcommand{\B}{\mathcal B}
\newcommand{\T}{T_\#}
\newcommand{\N}{\mathbb N}
\newcommand{\R}{\mathbb R}
\newcommand{\upr}{\mathbb{R}_<^2}

\begin{document}

\begin{titlepage}
    \centering
    %{\includegraphics[width=0.2\textwidth]{logo}\par}
    \vspace{1cm}
    {\bfseries\LARGE \institution \par}
    \vspace{1cm}
    {\scshape\Large Mathematic Analysis Fundamentals \par}
    \vspace{3cm}
    {\scshape\Huge \projecttitle \par}
    \vspace{3cm}
    {\itshape\Large End of Course Thesis.
    
    2024-2025. \par}
    \vfill
    {\Large Author: \par}
    {\Large \authorname \par}
    \vfill
    {\Large January 2025 \par}
\end{titlepage}


\setlength{\parskip}{0.75em}
\renewcommand{\baselinestretch}{1.25}


\subsection*{Abstract}
In this thesis we introduce the optimal transport problems from Monge and Kantorovich formulations, in order to introduce the Wasserstein distance for probability measures. Then, we re introduce the distance for persistence diagrams and prove that there exists an isometric embedding from separable, bounded metric spaces into the space of persistence diagrams with the bottleneck distance.

\subsection*{Key words}
Optimal transport, Wasserstein distance, bottleneck distance, TDA.

\tableofcontents
\pagebreak

\section{Introduction}

Transport maps were itroduced in 1781 by Gaspard Monge to represent the idea of moving earth from one place into an other \cite{Figalli}[1.1 Historical overview]. In this original formulation of the optimal transport problem, it was enough to consider $ \mathbb R^3 $ as the ambient space, using the Euclidean distance as the cost function of moving mass between two points.

In the 30's, Leonid Kantorovich reformulated the problem to describe the optimization process of supply and demand distributions of diverse problems. The mass could be divided between different origin and destinations, making it possible to interpret the problem as the way to measure the cost of transforming one probability distribution into an other. In this thesis, will introduce the $p$-Wasserstein distance as a metric on the probability measures with finite $p$-moment space. When $ p = 1 $, the distance will represent the metric introduced in the Kantorovich optimal transport problem, also used named Earth Mover's distance, used for machine learning algorithms and computer vision problems \cite{earth}. When $p = \infty $ it is named the bottleneck distance, and will be the main them of study of this thesis.

In topological data analysis, diagrams arise to represent the 
persistence of the homology groups of a data set through time. Those diagrams are named persistence diagrams, and those homology groups, persistence homology groups. We will introduce an analogous $p$-Wasserstein distance in the space of persistence diagrams and prove that there exists an isometric embedding  from a separable metric space into the space of persistence diagrams with the Wasserstein distance.
\pagebreak

\section{Optimal transport}
The main result of optimal transport theory is the solution of Kantorovich's problem for general costs, the existence of an optimal transport plan.

\begin{proposition}
    Let $ c: X \times Y \to [0, \infty] $ be lower semicontinous, and let $ \mu \in \mathcal P(X) $ and $ \nu \in \mathcal P(Y) $. Then there exists a coupling $ \bar \gamma \in \Gamma(\mu, \nu) $ that verifies
    $$
        \bar \gamma = \min \left\{ \gamma \in \Gamma(\mu, \nu) \ : \ \int_{X \times Y} c(x, y) d\gamma(x,y) \right\}.
    $$
\end{proposition}
\begin{proof}
    (to do)
\end{proof}

We will denote the set of probability measures over a space X by $ \mathcal P(X) $. 

\begin{example}[Mean and variance in $ \mathbb R $]
    
\end{example}

\begin{definition}
    Let $ (X, d) $ be a locally compact and separable, metric space. Let $ 1 \leq p < \infty $. The {\it set of probability measures with finite $p$-moment} is defined As
    $$
        \mathcal P_p (X) := \left\{ \sigma \in \mathcal P (X) : \int_X d(x, x_0)^p d \mu (x) < \infty \text{ for some } x_0 \in X \right\}.
    $$
\end{definition}

\begin{proposition}
    The definition of $ \mathcal P_p (X) $ is independent of the base point $ x_0 $
\end{proposition}
\begin{proof}
    (to do)
\end{proof}

\begin{definition}[$p$-Wasserstein distance]
    Given $ u, v \in \mathcal P_p (X) $, the {\it $p$-Wasserstein distance} is defined as
    $$
        W_p(u, v) := \left( \inf_{\gamma \in \Gamma(u, v)} \int_{X \times X} d(x,y)^p d\gamma(x, y)\right)^{\frac{1}{p}}.
    $$
\end{definition}

\begin{proposition}
    $W_p$ is a distance on the space $ \mathcal P_p(X) $.
\end{proposition}

\begin{proof}
    We will follow the steps made in \cite{Figalli}[Theorem 3.1.5].
    To prove the triangle inequality, let $ \mu_1, \mu_2, \mu_3 \in \mathcal P_p(X) $ and 

    (to do)
\end{proof}
\pagebreak

\section{Wasserstein distance over persistence diagrams}

The contents of this thesis are based on \cite{Figalli} and \cite{Bubenik}.

Along this text, we will denote the strict upper triangular region of the Euclidean plane as $ \upr := \{(x, y) \in \mathbb R^2 : x < y\} $, and the diagonal of the plane as $ \Delta := \{(x, y) \in \mathbb R^2 : x = y\}$.

\begin{definition}[Persistence diagram]
    Let $ I $ be a countable set. A {\it persistence diagram} is a function $ D: I \to \upr $.
\end{definition}

\begin{definition}[Chebyshev distance](To do)
    $d_\infty := \max \{|a_x - b_x|, |a_y - b_y|\}$
\end{definition}

\begin{proposition}
    If $ a \in \upr $, then $ d_\infty(a, \Delta) = \inf_{t \in \Delta} d_\infty(a, t) = \frac{a_y - a_x}{2} $.
\end{proposition}
\begin{proof}
    (to do)  
\end{proof}

\begin{proposition}
    The upper triangular region of the Euclidean plane with the Chebyshev distance  $ (\upr, d_\infty) $ is a metric space.
\end{proposition}
\begin{proof}
    (To do)
\end{proof}

\begin{definition}[Partial matching]
    Let $ D_1: I_1 \to \upr $ and $ D_2: I_2 \to \upr $ be persistence diagrams. A {\it partial matching} between $ D_1 $ and $ D_2 $ is the triple $ (I_1', I_2', f) $ such that $ f: I_1' \to I_2' $ is a bijection with $ I_1' \subseteq I_1 $ and $ I_2' \subseteq I_2 $.
\end{definition}

\begin{definition}
    Let $ D_1: I_1 \to \upr $ and $ D_2: I_2 \to \upr $ be persistence diagrams. Let $ (I_1', I_2', f) $ be a partial matching between them. If $ p < \infty $, the {\it $p$-cost of $ f $} is defined as
    \begin{align*}
        \costp(f) := \bigg(&\sum_{i \in I_1'} d_\infty(D_1(i), D_2(f(i)))^p \\
        &+ \sum_{i \in I_1 \setminus I_1'} d_\infty(D_1(i), \Delta)^p \\
        &+ \sum_{i \in I_2 \setminus I_2'} d_\infty(D_2(i), \Delta)^p \bigg)^{\frac{1}{p}}.
    \end{align*}
    For $ p = \infty $, the {\it $\infty$-cost of $ f $} is defined as
    \begin{align*}
        \costi(f) := \max \bigg\{&\sup_{i \in I_1'} d_\infty(D_1(i), D_2(f_i)), \\
        &\sup_{i\in I_1 \setminus I_1'} d_\infty(D_1(i), \Delta), \\
        &\sup_{i\in I_2 \setminus I_2'} d_\infty(D_2(i), \Delta)\bigg\}.
    \end{align*}
\end{definition}

\begin{definition}[p-Wasserstein distance] \label{def-Wasserstein}
    Let $ D_1, D_2 $ be persistence diagrams. Let $ 1 \leq p \leq \infty $. Define
    $$
        \tilde \wdp (D_1, D_2) = \inf \{\costp(f) : f \text{ is a partial matching between } D_1 \text{ and } D_2 \}.
    $$
    Let $ \emptyset $ denote the unique persistence diagram with empty indexing set. Let $ (\dgmp, \wdp) $ be the space of persistence diagrams $ D $ that satisfy $ \tilde \wdp(D, \emptyset) < \infty $ modulo the equivalence relation $ D_1 \sim D_2 $ if $ \tilde \wdp (D_1, D_2) = 0 $. The metric $ \wdp $ is called the {\it $p$-Wasserstein distance}.
\end{definition}

\begin{definition}[Bottleneck distance]
    In the conditions of Definition \ref{def-Wasserstein}, if $ p = \infty $, the metric $ \wdi $ is called the {\it bottleneck distance}.
\end{definition}

\begin{proposition} \label{prop-empty-mathing-distance}
    There is only one matching between $ D: I \to \upr $ and $ \emptyset $. Hence,
    $$
        \tilde \wdp(D, \emptyset) = \left(\sum_{i\in I} d_\infty(D(i), \Delta)^p\right)^{\frac{1}{p}}.
    $$
\end{proposition}
\begin{proof}
    (To do)
\end{proof}

\begin{proposition}
    The space of persistence diagrams with the $p$-Wasserstein distance $(\dgmp, \wdp)$ is indeed a metric space.
\end{proposition}
\begin{proof}
    (To do)
\end{proof}

\begin{definition}[Isometric embedding]
    Let $ (X, d_X), (Y, d_Y) $ be metric spaces. An {\it isometric embedding} $ \eta: (X, d_X) \to (Y, d_Y) $ is a mapping that satisfies
    $$
        d_X(x_1, x_2) = d_Y(\eta(x_1), \eta(x_2))
    $$
    for all $x_1, x_2 \in X$.
\end{definition}

\begin{definition}[Ball]
    Let $ 1\leq p \leq \infty $. Let $ D_0 \in \dgmp $. The {\it ball} at the space of persistence diagrams is defined as $ B_p(D_0, r) := \{D \in \dgmp : w_p(D, D_0) < r \} $.
\end{definition}

\begin{theorem}[Isometric embeeding of metric spaces into persistance diagrams]
    Let $ (X, d) $ be a separable, bounded metric space. Then there exists an isometric embedding to the space of persistence diagrams $ \eta: (X, d) \to (\dgmi, \wdi)$ such that $ \eta(X) \subseteq B(\emptyset, \frac{3c}{c}) \backslash B(\emptyset, c) $.
\end{theorem}

\begin{proof}
    As $ (X, d) $ is bounded, we can let $ c > \sup \{d(x, y): \  x, y \in X\} $. As $ (X, d) $ is separable, we can take $ \{x_k\}_{k=1}^\infty $, a countable, dense subset of $ (X, d) $. Consider
    \begin{align*}
        \eta: (X, d) &\to (\dgmi, \wdi) \\
        x &\mapsto \{(2c(k-1), 2ck + d(x, x_k))\}_{k=1}^\infty
    \end{align*}
    For any $ x \in X $ and $ k \in \n$,
    \begin{align*}
        d_\infty((2c(k-1), 2ck + d(x, x_k)), \Delta) 
        &= \frac{2ck + d(x, x_k) - 2c(k-1)}{2} \\
        &= c + \frac{d(x, x_k)}{2} \\
        &< c + \frac{c}{2} = \frac{3c}{2}.
    \end{align*}
    Because of Proposition \ref{prop-empty-mathing-distance}, for every $ x \in X $, $ \tilde \wdi(\eta(x), \emptyset) < \infty $ and $ \eta $ is well defined. Note that
    \begin{align*}
        \wdi(\eta(x), \emptyset) = \sup_{1 \leq k < \infty} d_\infty((2c(k-1), 2ck + d(x, x_k)), \Delta),
    \end{align*}
    so $ \eta(x) \in B(\emptyset, \frac{3c}{c}) \backslash B(\emptyset, c)$.

    Let $ \eta(x) $ and $ \eta(y) $ two equivalence classes of $ (\dgmi, \wdi) $. Choose the representative diagrams $D_x : \n \to \upr $ and $D_y : \n \to \upr $ and consider the partial matching $ (\n, \n, \operatorname{id}_\n) $. With it, for every $ k \in \n $, $ (2c(k-1), 2ck + d(x, x_k)) $ is matched with $ (2c(k-1), 2ck + d(y, x_k)) $. The Chebyshev distance between those points is
    \begin{align*}
        d_\infty(D_x(k), D_y(k)) 
        &= \max \big\{|2c(k-1) - 2c(k-1)|, \\
        &\quad |2ck + d(x, x_k) - b_y - (2ck + d(y, x_k))| \big\} \\
        &= \max \{0, |d(x, x_k) - d(y, x_k)|\} \\
        &= |d(x, x_k) - d(y, x_k)|.
    \end{align*}
    Hence, because of the triangle inequality, the cost of this partial matching is
    \begin{align*}
        \costi(\operatorname{id}_\n) = \sup_k |d(x, x_k - d(y, x_k)) \leq d(x, y).
    \end{align*}
    Since $ \{ x_k \}_{k=1}^\infty $ is dense, for every $ \epsilon > 0 $, there exist a $ k \in \n $ such that $ d(x, x_k) \leq \epsilon $, so
    \begin{align*}
        |d(x, x_k) - d(y, x_k)| 
        &\geq d(y, x_k) - d(x, x_k) \\
        &= d(y, x_k) + d(x, x_k) - d(x, x_k) - d(x, x_k) \\
        &\geq d(x, y) - 2d(x, x_k) \\
        &> d(x, y) - 2\epsilon.
    \end{align*}
    Therefore, $ \sup_k |d(x, x_k - d(y, x_k)) \geq d(x, y) $ and
    \begin{align*}
        \costi(\operatorname{id}_\n) = \sup_k |d(x, x_k - d(y, x_k)) = d(x, y).
    \end{align*}
    Suppose $ I, J \subseteq \n $ and $ (I, J, f) $ is a different partial matching between $ D_x $ and $ D_y $. Then there exist a $ k \in \n $ such that either $ k \notin I $ or $ k \in I $ and $ f(k) = k \neq k $. If $ k \notin I $, then
    \begin{align*}
        \costi(f) \geq d_\infty((2c(k-1), 2ck + d(x, x_k)), \Delta) \geq c.
    \end{align*}
    If $ k \in I$ and $ f(k) = k \neq k $, then
    \begin{align*}
        \costi(f) \geq || (2c(k-1), 2ck + d(x, x_k)) - (2c(k'-1), 2ck' + d(x, x_{k'}))||_\infty \geq 2 \epsilon.
    \end{align*}
    Hence, $ \costi(f) \geq c > d(x, y)$ and $d(x,y) = \wdi(\eta(x), \eta(y)) $, proving that $ \eta $ is an isometric embedding of a metric space into the space of persistence diagrams.
\end{proof}
\pagebreak

\bibliographystyle{acm}
\bibliography{biblio.bib}

\end{document}
