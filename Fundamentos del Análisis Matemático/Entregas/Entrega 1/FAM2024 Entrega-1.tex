%%%%%%%%%%%%%%%%%%

\documentclass[11pt,a4paper,draft]{article}

%%%%%%%%%%%%%%%%%%

\usepackage{latexsym}
\usepackage{amssymb}
\usepackage{amsmath}
\usepackage{amsfonts}
\usepackage{graphicx}
\usepackage{epsfig}
\usepackage{epstopdf}
\usepackage{esint}

\usepackage{amsthm}

\usepackage{mathtools} %for \stackrel

%%%%%%%%%%%%%%%%%%

\usepackage{color}
\newcommand{\red}{\textcolor{red}}
\newcommand{\blue}{\textcolor{blue}}

%%%%%%%%%%%%%%%%%%

\usepackage[spanish]{babel}
\usepackage[utf8]{inputenc}

%%%%%%%%%%%%%%%%%%

\addtolength{\topmargin}{-3.5cm} \addtolength{\oddsidemargin}{-2cm}
\addtolength{\textheight}{+5cm} \addtolength{\textwidth}{+4cm}


\parindent=0mm
\parskip 0mm

\thispagestyle{empty}


%%%%%%%%%%%%%%%%%%%%%%%%%%%%%%%%%%%
%%%%%%%%%%%%%%%%%%%%%%%%%%%%%%%%%%%


\begin{document}
\hfill (\red{\it A entregar el 26.09.2024})
\hrule\hrule
\vspace{1mm}

\noindent {\bf Fundamentos de Análisis Matemático, MMA 2023-24.
\hfill{Entrega 1}}

\vspace{1mm}

 \noindent {\bf Nombre y apellidos}: Gonzalo Ortega Carpintero
\vspace{2mm}

\hrule\hrule

\vspace{2mm}

\

{\bf 1.} Se dice que la función $f:\mathbb R^N \to \mathbb R$ es H\"older de orden $\alpha>0$ si existe una constante $C$ de forma que
$$
|f(x)-f(y)|\le C|x-y|^\alpha, \; \forall x,y \in \mathbb R^N. 
$$
Probar que si $f$ es H\"older de orden $\alpha$, con $\alpha>1$, entonces $f$ es constante. 

\vskip 5mm

 {\bf 2.}  En $\mathbb R^N$, si el conjunto $A$ no es medible Lebesgue, y  $s<N$, probar que  $\mathcal{H}^s(A)=\infty$.
 
\vskip 5mm

 \noindent {\bf 3.} Recordamos que la medida exterior de Lebesgue se define como
 \vskip -3mm
 $$
 m^*(E)=\inf\left\{\sum_{j\ge 1} \mbox{vol}(B_j): \{B_j\}_j \mbox{ cubrimiento por bolas de } A\right\}. 
 $$
 \vskip -2mm
  Definimos por otro lado la clase
  $$
  \mathcal{B}=\{A\subset \mathbb R^N: \forall \epsilon>0, \; \exists \mathcal{O},  \mbox{ abierto, tal que }\; A\subset \mathcal{O} \mbox{ y } \; m^*(\mathcal{O}\setminus A)<\epsilon\}. 
  $$
Probar: 
\begin{itemize}
\item  $\mathcal{B}$ es una $\sigma$-\'algebra en $\mathbb R^N$.
\item  $\mathcal{B}$ coincide con la $\sigma$-\'algebra $\mathcal A$ obtenida por el teorema de Caratheodory. 
\end{itemize}

\vskip 1mm
\hrule
\vskip 3mm

\noindent {\bf SOL.:} 
 
\vskip 5mm
{\bf 1.}
\begin{proof}
  Dados $x, y \in \mathbb{R}^N $ cualesquiera consideramos el segmento $ \left[xy\right] \subset \mathbb{R}^N $. Podemos dividir dicho segmento en $ n $ subsegmentos de la forma  $ \left[x_{i-1} x_i\right] $ con $ i \in \left[1, n\right] $ y $ x_0 = x $, $ x_n = y$, que cumplan
$$
  |x_i - x_{i-1}| = \frac{|x-y|}{n}.
$$

Para cada subsegmento se tiene que verificar la propiedad de ser Hölder, luego
\begin{align*}
  |f(y) - f(x)| &= \left|\sum_{i=1}^n (f(x_i) - f(x_{i-1}))\right| \leq \sum_{i=1}^n \left|(f(x_i) - f(x_{i-1}))\right| \\
  & \leq \sum_{i=1}^n C |x_i - x_{i-1}|^\alpha = n C \frac{|x-y|^\alpha}{n^\alpha} = C \frac{|x-y|^\alpha}{n^{\alpha-1}} \underset{\underset{\alpha>1}{n\to\infty}}{\longrightarrow} 0.
\end{align*}
Por tanto  $ \forall x, y \in \mathbb{R}^N, f(x) = f(y) $, teniendo que ser $ f $ una función constante.
\end{proof}

\newpage
{\bf 2.}
\begin{proof}
  Supongamos que la medida de Hausdorff es finita $ \mathcal H^s < \infty$, $ s<N $.
  Si desarrollamos ahora tenemos que
  \begin{align*}
    \mathcal H^N(A) &= \lim_{\delta \to 0^+} \operatorname{inf} \left\{\sum_j (\operatorname{diam} E_j)^N: A \subset \bigcup_{j=1}^\infty E_j, \; \operatorname{diam} E_j \leq \delta, \; \forall j \right\} \\
    &= \lim_{\delta \to 0^+} \operatorname{inf} \left\{\sum_j (\operatorname{diam} E_j)^{s+\epsilon}: A \subset \bigcup_{j=1}^\infty E_j, \; \operatorname{diam} E_j \leq \delta, \; \forall j \; \epsilon > 0 \right\} \\
    &\leq \lim_{\delta \to 0^+} C \delta^\epsilon *
    \lim_{\delta \to 0^+} \operatorname{inf} \left\{\sum_j (\operatorname{diam} E_j)^s: A \subset \bigcup_{j=1}^\infty E_j, \; \operatorname{diam} E_j \leq \delta, \; \forall j \; \right\}\\
    &= \lim_{\delta \to 0^+} C \delta^\epsilon * \mathcal H^s(A) = 0
  \end{align*}

  para alguna constante $ C $. Pero sabemos de clase que $ \mathcal H^N(A) = C_n m^*(A) $ con $ C_n \in \mathbb R^N $ constante y $ m^*(A) $ la medida de Lebesgue en $ \mathbb R^N $. Por tanto se tendría $ m^*(A) = 0 $ y $ A $ sería medible Lebesgue, entrando en contradicción. Por tanto, tiene que ser $ \mathcal H^s = \infty$.
\end{proof}

\vskip 10mm
{\bf 3.}
\begin{proof}
  Para probar que $ \mathcal B $ es una $ \sigma $-álgebra, basta comprobar que coincide con la $ \sigma $-álgebra $ \mathcal A $ obtenida por el teorema de Caratheodory.

  \vskip 5mm
  Para ello, empezamos tomando $ A \in \mathcal A $, por lo que 
  $
    \forall E \in \mathbb R^N, m^*(E \cap A) + m^*(E \cup A^c).
  $
  Definimos
  $
    \mathcal O_\epsilon = \left\{ x \in \mathbb R^N: d(x, A) < \epsilon\right\},
  $
  conjunto abierto que satisface $ A \subset \mathcal O_\epsilon $. Se tiene entonces que
  $$
    m^*(\mathcal O_\epsilon) = m^*(\mathcal O_\epsilon \cap A) + m^*(\mathcal O_\epsilon \cup A^c) = m^*(A) + m^*(\mathcal O_\epsilon \setminus A).
  $$
  Despejando obtenemos $m^*(\mathcal O_\epsilon \setminus A) = m^*(\mathcal O_\epsilon) - m^*(A) \underset{\epsilon \to 0}{=} 0 $. Luego $ A \in \mathcal B $. 

  \vskip 5mm
    Tomando ahora $ B \in \mathcal B $, se tiene entonces que 
    $
      \forall \epsilon>0, \; \exists \mathcal{O},  \mbox{ abierto, tal que }\; B\subset \mathcal{O} \mbox{ y } \; m^*(\mathcal{O}\setminus B)<\epsilon,
    $
    y tomando un conjunto $ E \in \mathbb R^n $ cualquiera y usando que la medida de Lebesgue es una medida exterior tenemos
    \begin{align*}
      m^*(E \cap B) + m^*(E \cap B^c) &= m^*(E \cap B) + m^*((E \cap \mathcal O^c) \cup (E \cap (\mathcal O \setminus B)) \\
      &\leq m^*(E \cap \mathcal O) + m^*(E \cap \mathcal O^c) + m^*(E \cap (\mathcal O \setminus B)) \\
      &\leq m^*(E \cap \mathcal O) + m^*(E \cap \mathcal O^c) + \epsilon \leq m^*(E) + \epsilon,
    \end{align*}
  puesto que al ser $ \mathcal O $ abierto, $m^*(E \cap \mathcal O) + m^*(E \cap \mathcal O^c) = m^*((E \cap \mathcal O) \cup (E \cap \mathcal O^c))$. Como $ \epsilon $ puede ser tan pequeño como se quiera se tiene entonces la condición suficiente para que ser medible Lebesgue
  $
    m^*(E) = m^*(E \cap B) + m^*(E \cap B^c),
  $
  y por tanto $ B \in \mathcal A $.

  \vskip 5mm

\end{proof}
\end{document}
