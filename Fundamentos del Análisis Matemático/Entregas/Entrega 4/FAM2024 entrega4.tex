\documentclass[11pt,a4paper]{article}

%%%%%%%%%%%%%%%%%%

\thispagestyle{empty}

\usepackage{latexsym}
\usepackage{amssymb}
\usepackage{amsmath}
\usepackage{amsfonts}
\usepackage{color}
\usepackage{graphicx}
\usepackage{epsfig}
\usepackage{epstopdf}
\usepackage{verbatim}
\usepackage{mathabx}
\usepackage{mathrsfs}
\usepackage{amsthm}

%%%%%%%%%%%%%%%%%%

\usepackage[spanish]{babel}
\usepackage[utf8]{inputenc}

\newcommand{\blue}{\textcolor{blue}}
\newcommand{\red}{\textcolor{red}}
\newcommand{\w}{\textcolor{white}}


%%%%%%%%%%%%%%%%%%

\addtolength{\topmargin}{-3cm} 
\addtolength{\oddsidemargin}{-2.5cm}
\addtolength{\textheight}{+6.5cm} 
\addtolength{\textwidth}{+5cm}


\parindent=0cm    
\parskip=0mm

%%%%%%%%%%%%%%%%%%



\begin{document}
\hrule\hrule
\vspace{2mm}

{\bf Fundamentos de Análisis Matemático, MMA 2024-25.
\hfill{ENTREGA 3}}

\vspace{3mm}

 \noindent {\bf NOMBRE}: Gonzalo Ortega Carpintero \hfill ({\small \it Para entregar el jueves, 5 de diciembre})

\vspace{2mm}

\hrule\hrule

\vspace{2mm}

\

{\bf 1.-} (Lema de Van der Corput) Sea $\Phi\in\mathcal{C}^2(\mathbb R)$ una función a valores reales, donde $ \Phi'(x) $ es monótona y cumple $|\Phi'(x)|\ge 1$ en el intervalo $[a,b]$. Demuestra que
$$
\left|\int_a^b e^{i\Phi(x)}dx\right|\le 4. 
$$

 \vskip -1mm

{\sc Indicaciones}: Escribe $e^{i\Phi(x)}=\Phi'(x)e^{i\Phi(x)}\frac 1{\Phi'(x)}$ y usa integración por partes. Luego observa que $\frac d{dx}\left(\frac 1{\Phi'(x)}\right)$ no cambia de signo por ser $\Phi'(x)$ monótona. 

\vskip 8mm 

{\bf 2.-} Demuestra que existe una constante $C>0$, finita, de forma que $\forall \lambda\in \mathbb R$ y $\forall a<b$ se tiene
$$
\left|\int_a^b e^{i(\lambda x^2+x)}\frac{dx}{|x|^{1/2}}\right|\le C
$$
 \vskip -1mm

{\sc Indicaciones}: Podemos suponer que $0 \le a <b$. Haz un cambio de variables para que la integral quede de la forma $\displaystyle \int_{a'}^{b'} e^{i\Phi(y)}dy$. Finalmente, encuentra las regiones de monotonía de $\Phi'(x)$ y donde, además, $|\Phi'(x)|\ge 1$. Deberás considerar los casos $\lambda>0$ (fácil) y $  \lambda<0$ por separado. 
\vskip 1cm 

\noindent {\bf SOL.:} 
 
\section*{1.}
\begin{proof}
Siguiendo la indicación e integrando por partes se tiene:
\begin{align*}
    \left|\int_a^b e^{i\Phi(x)} \ dx\right| &= \left|\int_a^b i \Phi'(x) e^{i\Phi(x)} \frac 1{i \Phi'(x)} \ dx\right| = \left| \left[ \frac{-i}{\Phi'(x)} e^{i\Phi(x)}  \right]_a^b - \int_a^b e^{i\Phi(x)} \frac{i \Phi'(x)}{\Phi''(x)^2} \ dx\right| \\ 
    &= \left| \frac{-i}{\Phi'(b)} e^{i\Phi(b)} - \frac{-i}{\Phi'(a)} e^{i\Phi(a)} - \int_a^b e^{i\Phi(x)} \frac{-i\Phi''(x)} {\Phi'(x)^2} \ dx\right| \\
    &\le \left| \frac{-i}{\Phi'(b)} e^{i\Phi(b)} \right| + \left| \frac{-i}{\Phi'(a)} e^{i\Phi(a)} \right| + \left| \int_a^b e^{i\Phi(x)} \frac{-i\Phi''(x)} {\Phi'(x)^2} \ dx \right| \\
    &\le \left| \frac{1}{\Phi'(b)} \right| + \left| \frac{1}{\Phi'(a)} \right| + \int_a^b \left|\frac{\Phi''(x)} {\Phi'(x)^2} \right| \ dx \le 1 + 1 + \int_a^b|\Phi''(x)| \ dx \\
    &\le 2 + |\Phi'(b)| + |\Phi'(a)| \le 2 + 1 + 1 = 4,
\end{align*}
usando, en las dos últimas desigualdades, el hecho de $|\Phi'(x)|\ge 1$ y que $ \Phi'(x) $ no cambia de signo.
\end{proof}

\newpage
\section*{2.}
\begin{proof}
Para $ 0 \leq a \le b $, podemos tomar $ |x| = x $ haciendo el cambio de variable $ y = x^{\frac{1}{2}} $, tenemos $ x = y^2 $ nos queda $\Phi(y) = \lambda y^4 + y^2 $, con $ dx = 2y \ dy $. Luego
$$
\left|\int_a^b e^{i(\lambda x^2+x)}\frac{dx}{|x|^{1/2}}\right| = \left|\int_{\sqrt a}^{ \sqrt b} e^{i(\lambda y^4 + y^2)}dy\right|.
$$
Basta por tanto acotar la integral cuando $ \Phi(y) $ no cumple las hipótesis del Ejercicio 1. Tenemos $ \Phi'(y) = 4\lambda y^3 + 2y$ y $ \Phi''(y) = 12\lambda y^2 + 2$.

\vskip 5mm 
Para $ \lambda > 0 $, $\Phi''(y)$ es siempre positiva y por tanto $ \Phi'(y) $ siempre monótona al ser creciente. En caso de que $ |\Phi'(y)| \ge 1 $ la integral estaría acotada por el ejercicio anterior. Para $ y > 1 $ se cumple dicha desigualdad, y basta ver que $|e^{i\Phi(y)}| = 1 $, luego el tramo $ \int_{0}^{1} |e^{i\Phi(y)}| \ dy $ está acotado también. Faltaría evaluar el caso $ \lambda < 0 $.

\end{proof}
\end{document}
