\documentclass[11pt,a4paper]{article}

%%%%%%%%%%%%%%%%%%

\thispagestyle{empty}

\usepackage{latexsym}
\usepackage{amssymb}
\usepackage{amsmath}
\usepackage{amsfonts}
\usepackage{color}
\usepackage{graphicx}
\usepackage{epsfig}
\usepackage{epstopdf}
\usepackage{verbatim}
\usepackage{mathabx}
\usepackage{mathrsfs}

%%%%%%%%%%%%%%%%%%

\usepackage[spanish]{babel}
\usepackage[utf8]{inputenc}

\newcommand{\blue}{\textcolor{blue}}
\newcommand{\red}{\textcolor{red}}
\newcommand{\w}{\textcolor{white}}


%%%%%%%%%%%%%%%%%%

\addtolength{\topmargin}{-3cm} 
\addtolength{\oddsidemargin}{-2.5cm}
\addtolength{\textheight}{+6.5cm} 
\addtolength{\textwidth}{+5cm}


\parindent=0cm    
\parskip=0mm

%%%%%%%%%%%%%%%%%%



\begin{document}
\hrule\hrule
\vspace{2mm}

{\bf Fundamentos de Análisis Matemático, MMA 2024-25.
\hfill{ENTREGA 3}}

\vspace{3mm}

 \noindent {\bf NOMBRE}: Gonzalo Ortega Carpintero \hfill ({\small \it Para entregar el jueves, 5 de diciembre})

\vspace{2mm}

\hrule\hrule

\vspace{2mm}

\

{\bf 1.-} (Lema de Van der Corput) Sea $\Phi\in\mathcal{C}^2(\mathbb R)$ una función a valores reales, monótona y con $|\Phi'(x)|\ge 1$ en el intervalo $[a,b]$. Demuestra que
$$
\left|\int_a^b e^{i\Phi(x)}dx\right|\le 4. 
$$

 \vskip -1mm

{\sc Indicaciones}: Escribe $e^{i\Phi(x)}=\Phi'(x)e^{i\Phi(x)}\frac 1{\Phi'(x)}$ y usa integración por partes. Luego observa que $\frac d{dx}\left(\frac 1{\Phi'(x)}\right)$ no cambia de signo por ser $\Phi'(x)$ monótona. 

\vskip 8mm 

{\bf 2.-} Demuestra que existe una constante $C>0$, finita, de forma que $\forall \lambda\in \mathbb R$ y $\forall a<b$ se tiene
$$
\left|\int_a^b e^{i(\lambda x^2+x)}\frac{dx}{|x|^{1/2}}\right|\le C
$$
 \vskip -1mm

{\sc Indicaciones}: Podemos suponer que $0 \le a <b$. Haz un cambio de variables para que la integral quede de la forma $\displaystyle \int_{a'}^{b'} e^{i\Phi(y)}dy$. Finalmente, encuentra las regiones de monotonía de $\Phi'(x)$ y donde, además, $|\Phi'(x)|\ge 1$. Deberás considerar los casos $\lambda>0$ (fácil) y $  \lambda<0$ por separado. 
\vskip 1cm 

\noindent {\bf SOL.:} 
 
 \end{document}

%%%%%%%%%%%%%%%%%%%%%%%%%%%%%%%%%%%
%%%%%%%%%%%%%%%%%%%%%%%%%%%%%%%%%%%

