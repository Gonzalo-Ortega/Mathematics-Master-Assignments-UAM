\documentclass[11pt,a4paper]{article}

%%%%%%%%%%%%%%%%%%

\thispagestyle{empty}

\usepackage{latexsym}
\usepackage{amssymb}
\usepackage{amsmath}
\usepackage{amsfonts}
\usepackage{color}
\usepackage{graphicx}
\usepackage{epsfig}
\usepackage{epstopdf}
\usepackage{verbatim}
\usepackage{mathabx}
\usepackage{mathrsfs}

%%%%%%%%%%%%%%%%%%

\usepackage[spanish]{babel}
\usepackage[uft8]{inputenc}

\newcommand{\blue}{\textcolor{blue}}
\newcommand{\red}{\textcolor{red}}
\newcommand{\w}{\textcolor{white}}

\addtolength{\topmargin}{-3cm} 
\addtolength{\oddsidemargin}{-2.5cm}
\addtolength{\textheight}{+6.5cm} 
\addtolength{\textwidth}{+5cm}


\parindent=0cm    
\parskip=0mm

%%%%%%%%%%%%%%%%%%



\begin{document}
\hrule\hrule
\vspace{2mm}

\noindent {\bf Fundamentos de Análisis Matemático, MMA 2024-25.
\hfill{ENTREGA 3}}

\vspace{3mm}

 \noindent {\bf NOMBRE}: \hfill ({\small \it Para entregar el jueves, 21 de noviembre})

\vspace{2mm}

\hrule\hrule

\vspace{2mm}

\

\noindent  {\bf 1.-} 
 \blue{\em Demuestra que si $f\in\mathscr{S}(\mathbb R^N)$ y $\alpha\ge 0$ entonces la siguiente función pertenece a   $L^p(\mathbb R^N), \quad \forall p\ge 1$:
$$
G(x)=\left(\widehat f(\cdot)|\cdot|^\alpha\right)^{\widecheck{}}(x)=\int_{\mathbb R^N}
\widehat f(\xi)|\xi|^\alpha e^{2\pi ix\cdot\xi}d\xi.
$$}
\vskip -2mm
{\sc Indicaciones}: 
\begin{enumerate}
\item Justifica primero por qué es suficiente probarlo para  $p=1$ y $0<\alpha<2$. 
\item Demuestra que si $0<\alpha<2$ 
$$\displaystyle\int_{\mathbb R^N}\left(\int_{|y|\le 1}\frac{|f(x+y)-f(x)-\nabla f(x)\cdot y|}{|y|^{N+\alpha}}dy+
\int_{|y|>1}\frac{|f(x+y)-f(x)|}{|y|^{N+\alpha}}dy\right)dx<\infty.$$ 

\item  Lo anterior nos dice que la siguiente función  pertenece a $L^1(\mathbb R^N):$ ${}^($\footnote{ Puede ser útil observar que, al ser 
$\nabla f(x)\cdot y$ una función impar en la variable  $y$, se tiene $$F(x)=\displaystyle\int_{|y|\le R}\frac{f(x+y)-f(x)-\nabla f(x)\cdot y}{|y|^{N+\alpha}}dy+
\int_{|y|>R}\frac{f(x+y)-f(x)}{|y|^{N+\alpha}}dy, \quad \forall R>0.$$}${}^)$
 \\[2mm]
$F(x)=\displaystyle\int_{|y|\le 1}\frac{f(x+y)-f(x)-\nabla f(x)\cdot y}{|y|^{N+\alpha}}dy+
\int_{|y|>1}\frac{f(x+y)-f(x)}{|y|^{N+\alpha}}dy.$    \\[1mm]
Prueba ahora que $\widehat F(\xi)=C_0\,\widehat G(\xi)$, para una constante $C_0=C(N,\alpha)$ que depende solo de  $N$ y $\alpha$.
\end{enumerate}
\vskip 6mm
\hrule
\vskip 5mm


\noindent {\bf SOL.:} 
 
 \end{document}
