\documentclass[11pt,a4paper]{article}

%%%%%%%%%%%%%%%%%%

\thispagestyle{empty}

\usepackage{latexsym}
\usepackage{amssymb}
\usepackage{amsmath}
\usepackage{amsfonts}
\usepackage{color}
\usepackage{graphicx}
\usepackage{epsfig}
\usepackage{epstopdf}
\usepackage{verbatim}
\usepackage{mathabx}
\usepackage{mathrsfs}
\usepackage{amsthm}

%%%%%%%%%%%%%%%%%%

\usepackage[spanish]{babel}
\usepackage[utf8]{inputenc}

\newcommand{\blue}{\textcolor{blue}}
\newcommand{\red}{\textcolor{red}}
\newcommand{\w}{\textcolor{white}}

\addtolength{\topmargin}{-3cm} 
\addtolength{\oddsidemargin}{-2.5cm}
\addtolength{\textheight}{+6.5cm} 
\addtolength{\textwidth}{+5cm}


\parindent=0cm    
\parskip=0mm

%%%%%%%%%%%%%%%%%%



\begin{document}
\hrule\hrule
\vspace{2mm}

\noindent {\bf Fundamentos de Análisis Matemático, MMA 2024-25.
\hfill{ENTREGA 3}}

\vspace{3mm}

 \noindent {\bf NOMBRE}: Gonzalo Ortega Carpintero \hfill ({\small \it Para entregar el jueves, 21 de noviembre})

\vspace{2mm}

\hrule\hrule

\vspace{2mm}

\

\noindent  {\bf 1.-} 
 \blue{\em Demuestra que si $f\in\mathscr{S}(\mathbb R^N)$ y $\alpha\ge 0$ entonces la siguiente función pertenece a   $L^p(\mathbb R^N), \quad \forall p\ge 1$:
$$
G(x)=\left(\widehat f(\cdot)|\cdot|^\alpha\right)^{\widecheck{}}(x)=\int_{\mathbb R^N}
\widehat f(\xi)|\xi|^\alpha e^{2\pi ix\cdot\xi}d\xi.
$$}
\vskip -2mm
{\sc Indicaciones}: 
\begin{enumerate}
\item Justifica primero por qué es suficiente probarlo para  $p=1$ y $0<\alpha<2$. 
\item Demuestra que si $0<\alpha<2$ 
$$\displaystyle\int_{\mathbb R^N}\left(\int_{|y|\le 1}\frac{|f(x+y)-f(x)-\nabla f(x)\cdot y|}{|y|^{N+\alpha}}dy+
\int_{|y|>1}\frac{|f(x+y)-f(x)|}{|y|^{N+\alpha}}dy\right)dx<\infty.$$ 

\item  Lo anterior nos dice que la siguiente función  pertenece a $L^1(\mathbb R^N):$ ${}^($\footnote{ Puede ser útil observar que, al ser 
$\nabla f(x)\cdot y$ una función impar en la variable  $y$, se tiene $$F(x)=\displaystyle\int_{|y|\le R}\frac{f(x+y)-f(x)-\nabla f(x)\cdot y}{|y|^{N+\alpha}}dy+
\int_{|y|>R}\frac{f(x+y)-f(x)}{|y|^{N+\alpha}}dy, \quad \forall R>0.$$}${}^)$
 \\[2mm]
$F(x)=\displaystyle\int_{|y|\le 1}\frac{f(x+y)-f(x)-\nabla f(x)\cdot y}{|y|^{N+\alpha}}dy+
\int_{|y|>1}\frac{f(x+y)-f(x)}{|y|^{N+\alpha}}dy.$    \\[1mm]
Prueba ahora que $\widehat F(\xi)=C_0\,\widehat G(\xi)$, para una constante $C_0=C(N,\alpha)$ que depende solo de  $N$ y $\alpha$.
\end{enumerate}
\vskip 6mm
\hrule
\vskip 5mm


\noindent {\bf SOL.:} 
\begin{proof}
    \
    \subsubsection*{Indicación 1:}
    Supongamos que $ G(x) \in L^1$, entonces, por propiedades de la transformada de Fourier (\cite{Soria} Proposición 6.1), se tiene que $ \widehat G(x) \in L^\infty $, y en particular $ \widehat G(x) \in L^1$. Por tanto, se tiene también que $ \widehat{\widehat G}(x) \in L^\infty$. Pero por el Teorema de Inversión en $\mathscr{S}$ (\cite{Soria} Teorema 6.8), $ \widehat{\widehat G} (x) = \widehat G(-x)^{\widecheck{}} = \left(\widehat f(\cdot)|\cdot|^\alpha\right)^{\widecheck{}}(-x) = G(-x)$, donde haciendo el cambio de variable $ x = -x $, nos queda que $ G(x) \in L^\infty $. Con esto vemos que sería suficiente con probar el ejercicio para $ p = 1 $.
    \vskip 5mm
    Si fuera $ \alpha = 0 $, entonces $ G(x) = f(x) \in L^\infty $ ya que $ f \in \mathscr S$. Si fuera $ \alpha \ge 2 $, podríamos expresar $ |\xi|^\alpha = |\xi|^{2n + \alpha^*} = |\xi|^{2n} |\xi|^{\alpha^*}$ con $ 2n \in 2\mathbb N $ y $\alpha^* \in [0, 2)$. Por la propiedad de la derivada de la transformada (\cite{Soria} Corolario 6.4), tenemos que $ \widehat{D^{2n}f}(\xi) = (2 \pi i \xi)^{2n} \widehat f (\xi) \ge |\xi|^{2m} \widehat f (\xi)$. Por tanto tenemos
    $$
    \int_{\mathbb R^N}
    \widehat f(\xi)|\xi|^\alpha e^{2\pi ix\cdot\xi}d\xi = \int_{\mathbb R^N}
    \widehat f(\xi) |\xi|^{2n} |\xi|^{\alpha^*} e^{2\pi ix\cdot\xi}d\xi \le \int_{\mathbb R^N}
    \widehat{D^{2n}f}(\xi) |\xi|^{\alpha^*} e^{2\pi ix\cdot\xi}d\xi.
    $$
    Como $ D^{2n}f \in \mathscr S $, para probar que esta última integral está en $ L^\infty $ bastaría probar nuestro problema para $ \alpha \in (0, 2)$. 

    \newpage

    \subsubsection*{Indicación 2:}
    Veamos que cada una de las dos integrales indicadas converge. Para la primera utilizamos el desarrollo de Taylor de $ f(x + y) = f(x) + \nabla f(x) \cdot y + \frac{1}{2}H_f(x) |y|^2 + \mathcal O (|y|^3)$. Luego
    $$
    \int_{\mathbb R^N}\int_{|y|\le 1}\frac{|f(x+y)-f(x)-\nabla f(x)\cdot y|}{|y|^{N+\alpha}}dy dx \le
    \int_{\mathbb R^N}\int_{|y|\le 1}\frac{H_f(x) |y|^2}{|y|^{N+\alpha}}dy dx < \infty
    $$
    ya que $ \alpha \in (0, 2) $ y $ f \in \mathscr S $. Para la segunda integral basta con usar la forma de los elementos de $ \mathscr S $, así
    $$
    \int_{\mathbb R^N}\int_{|y|>1}\frac{|f(x+y)-f(x)|}{|y|^{N+\alpha}}dydx \le \int_{\mathbb R^N}\int_{|y|>1}\frac{|f(x+y)|+|f(x)|}{|y|^{N+\alpha}}dydx 
    $$
    $$
    \le \int_{\mathbb R^N} \frac{2}{|x|^{2N}}\int_{|y|>1}\frac{1}{|y|^{N+\alpha}|x+y|^{2N}}dydx < \infty.
    $$
    Por tanto tenemos que si $\alpha \in (2, 0)$ se tiene $F(x) \in L^1$ como queríamos comprobar.
    \subsubsection*{Indicación 3:}
    Utilizando la nota $ ^{(1)} $ observamos que la segunda integral tiende a cero cuando $ R $ tiende a infinito, por lo que podemos considerar únicamente la primera integral. Calculando su transformada aplicando el Teorema de Fubini, usando las propiedades de la transformada (\cite{Soria} Prop. 6.1 {\it iv)}, Prop. 6.4) y sacando factor común $ \widehat f(\xi) $, tenemos
    \begin{align*}
        \widehat F (\xi) &= \lim_{R\to\infty} \int_{\mathbb R^N} \left( \int_{|y|\le R}\frac{f(x+y)-f(x)-\nabla f(x)\cdot y}{|y|^{N+\alpha}}dy\right) e^{2\pi i x \cdot \xi} dx \\
        &= \lim_{R\to\infty} \int_{|y|\le R} \frac{1}{|y|^{N+\alpha}} \left( \int_{|y|\le R} ( f(x+y)-f(x)-\nabla f(x)\cdot y ) e^{2\pi i x \cdot \xi} dx \right) dy \\
        &= \lim_{R\to\infty} \int_{|y|\le R} \frac{1}{|y|^{N+\alpha}} \left( e^{2\pi i x \cdot |\xi|} \widehat f(\xi) - \widehat f(\xi) - 2\pi i |\xi| \widehat f(\xi)\cdot y \right) dy \\
        &= \widehat f(\xi) \lim_{R\to\infty} \int_{|y|\le R} \frac{1}{|y|^{N+\alpha}} \left( e^{2\pi i y \cdot |\xi|} - 1 - 2\pi i |\xi|\cdot y \right) dy.
    \end{align*}
    Haciendo el cambio de variable $ u = y|\xi| $, con $ dy = \frac{1}{|\xi|^N} du $ tenemos
    $$
        \widehat F (\xi) = \widehat f(\xi) \lim_{R\to\infty} \int_{|y|\le R} \frac{|\xi|^{N+\alpha}}{|u|^{N+\alpha}} \left( e^{2\pi i u} - 1 - 2\pi i u \right) \frac{1}{|\xi|^N} du = \widehat f(\xi)|\xi|^\alpha C(N, \alpha) = \widehat G(\xi) C_0.
    $$
    Por tanto, como por la { \bf Indicación 2} $ F(x) \in L^1 $ para $ \alpha \in (0, 2) $, entonces $ G(x) $ también. Con la justificación de la {\bf Indicación 1}, esto prueba que $ G(x) \in L^\infty $ como se pedía.
        
\end{proof}

\begin{thebibliography}{9}

    \bibitem{Soria}
    Fernando Soria,
    \textit{Curso de Variable Real},
    Universidad Autónoma de Madrid 2019.
    
  \end{thebibliography}

\end{document}
